\chapter{What topic models are not}
\label{ch:boundaries}

Although topic models have many and varied applications, it is important to be clear and realistic about what they are {\em not} appropriate for.


\section{Summarization}



\section{Topic Detection and Tracking}

Despite the similarity in terms, the meaning of ``topic'' in topic modeling is distinct from the meaning in many information retrieval settings.
There is a substantial and well-developed field of topic detection and tracking \cite{allan-02}.
In TDT, a topic is usually closer to an event or an individual story.
In contrast, topic models tend to identify more abstract latent factors.
For example, a TDT topic might include an earthquake in Haiti, whereas a topic model might represent the same event as a combination of topics such as Haiti, natural disasters, and international aid.

There has been some work on using topic models to detect emerging events by searching for changes in topic probability \cite{alsumait-08}.
But these methods tend to identify mainly the fact that an event has occurred, without necessarily identifying the specific features of that event.
Other work has found that more lexically specific methods than topic models are best for identifying memes and viral phrases \cite{leskovec-09}.
%Discovering and tracking topics in streams is a classic topic in
%information retrieval, starting with the Topic Detection and Tracking
%track at TREC (Allan). Recently, this is continued as temporal
%summarization, see Kedzie, McKeown & Diaz for an example. People have
%applied topic models here as well.

%Allan, James, ed. Topic detection and tracking: event-based
%information organization. Vol. 12. Springer Science & Business Media,
%2012.

%Kedzie, Chris, Kathleen McKeown, and Fernando Diaz. "Predicting
%salient updates for disaster summarization." Proceedings of the 53rd
%annual meeting of the ACL and the 7th international conference on
%natural language processing. 2015.



\section{Feature Induction}

%Finally, various kinds of geo-location recommendation / summarization
%lends itself to topic modeling (for example, see Kurashima et al) and
%ties to some of the works on social media.

%Kurashima, Takeshi, et al. "Geo topic model: joint modeling of user's
%activity area and interests for location recommendation." Proceedings
%of the sixth ACM international conference on Web search and data
%mining. ACM, 2013.


Jacob Eisenstein's twitter location model.



\jbgcomment{Should connect to your multilingual chapter \yhcomment{added.}}
