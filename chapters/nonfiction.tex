
\chapter{Non-Fiction}
\label{ch:nonfiction}

Studying history is about encountering the unexpected, often in contexts that seem familiar. We don't necessarily know how people in the past talked about particular issues, or how they organized their lives. Perhaps more dangerously, we assume that we know these things, and that our ancestors saw the world in the same way we do. Topic models give us a perspective that is interpretable but at the same time alien, based on patterns in documents and not on our own conceptions of how things should be.

A useful resource:
Clay Templeton, Topic Modeling in the Humanities: An Overview. 
http://mith.umd.edu/topic-modeling-in-the-humanities-an-overview/

\section{Newspapers}

Newman and Block \cite{newman-06} present an early example of topic modeling on historical newspapers.

Nelson (Mining the Dispatch, http://dsl.richmond.edu/dispatch/) studies topics in Civil War-era newspapers, including the Confederate paper of record, the Daily Dispatch, and the New York Times.

Yang, Torget, and Mihalcea \cite{yang-11-historical} model a collection of historical newspapers from Texas, finding that the significance of the pivotal battle of San Jacinto was established much earlier than historians had anticipated.

\section{Historical Records}

Miller \cite{miller-13} uses Chinese records to investigate the meaning of the word {\em zei}, or ``bandit'' in Qing dynasty China. The word by itself can imply several different forms of anti-social behavior, which are difficult to distinguish from word frequencies alone. A topic model uses contextual information to separate these effects.

This work is notable for using individual Chinese characters as the tokenization.

Cameron Blevins (http://www.cameronblevins.org/posts/topic-modeling-martha-ballards-diary/) models the diary of Martha Ballard, a revolutionary war-era midwife who recorded entries over 27 years. The model provides a useful way to discover connections between words and repeated discourses.

\section{Secondary literature}

The historical record of scholarship is a valuable source for intellectual history. Many users make use of the JStor ``Data for Research'' API.

Mimno \cite{Mimno-12b} studies a collection of Classics journals to detect changes in the field over the 20th century.

Goldstone and Underwood \cite{Goldstone-14} analyze the proceedings of the Modern Language Association to find shifts in focus in the field of English literature.