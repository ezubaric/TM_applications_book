

\chapter{Understanding Scientific Publications}
\label{ch:sci}

In Chapter~\ref{ch:nonfiction}, we how scholars use topic models to understand
non-fiction documents.  This chapter focuses on a particular subgenre of
non-fiction: scientific documents.  Scientific documents deserve their own
chapter because these documents are unique: they use very specialized
vocabulary, they are the vehicles for innovation, and they shape important
policy decisions.  We discuss each of these aspects in turn.

\begin{figure}
\includegraphics[width=\linewidth]{figures/sci_faking}
\caption{Using the appropriate language is a prerequisite for being
  part of a field (but not sufficient).  Topic models use this to
  automatically discover fields of study.}
\label{fig:faking}
\end{figure}

\paragraph{Specialized Vocabularies Define Fields of Study}

First, scientific documents are unique because unlike ``general'' documents,
their vocabulary is precise and carefully measured.  ``Resistance'', ``splice'',
``utilization'', and ``demand'' are common words with radically different
meanings when used in specialized, technical contexts.  Their use is a
shibboleth whose use shows that you a member of a specific discipline
(Figure~\ref{fig:faking}).  Thus, the ability of topic models to capture
patterns of word usage also captures community and affiliation; this goes well
beyond the thematic uses of topic models described in previous chapters.

\paragraph{Scientific Documents Innovate}

As any researcher will tell you, not every scientific publication is innovative;
the sad truth is that most are not.  However, some scientific monographs are
Earth shattering (hopefully just figuratively).  Unlike the other domains we've
discussed, scientific documents are not just \emph{reports} of news or events;
they actually \emph{are the news}.

What makes the analysis of scientific document collections both challenging and
interesting is that innovation is hard to detect and hard to attribute.
Einstein's groundbreaking 1905 papers were not fully recognized until many years
later; important ideas are often proposed by an obscure researcher but only accepted
once popularized and supported by another research; which document (researcher)
in this case was the true source of the innovation?  As we will see in this
chapter, topic models can help answer this question.

\paragraph{Policy Makers Understanding Scientific Documents}

Understanding scientific publications is important for funding agencies,
lawmakers, and the public.  Government funding of science can create jobs,
improve culture, and is an important form of international ``soft power''.
However, knowing which research to fund is difficult, as the nature of science
means that fields constantly change, which precludes rigid
classifications~\citep{szostak-04}.  One challenge of modeling scientific
documents is modeling how fields change over time; the static models we've
discussed thus far are not always appropriate.

\section{Understanding Fields of Studies}
\label{sec:sci_fields}

\begin{figure}
  \includegraphics[width=.8\linewidth]{figures/sci_gs}
  \caption{After running a topic model on \abr{pnas},
    \citet{griffiths-04} found topics ($x$-axis) that could recreate
    the manually defined fields of study covered by \abr{pnas}
    ($y$-axis).}
\label{fig:pnas}
\end{figure}

One of the first uses of topic models was to understand the ``fields
of science''.  \citet{griffiths-04} found that they were able to
reconstruct the official \abr{pnas} topic codes automatically using
topic models (Figure~\ref{fig:pnas}).  This is a useful sanity check
that yes, indeed, topic models correlate with what we often think of
as scientific disciplines: they use distinct language for methods,
subjects of study, and have different key players.

However, Griffiths and Styvvers were not in a position to do anything
with their understanding of the fields of science.  In contrast,
\citet{talley-11} sought to understand the American National Institutes
of Health's funding priorities from within the organization.

The National Institutes of Health (\abr{nih}) are America's premiere funding
agency for biological and health research.  The \abr{nih} consists of
several institutes that focus on particular diseases, research techniques, or body
systems; each of these institutes manages its own independent funding portfolio,
sometimes making it difficult to understand the ``big figure'' of funding.

\citet{talley-11} used topic models to help create this big picture,
in contrast to more labor intensive techniques (e.g., keywords from a
meticulously organized ontology).  Their analysis discovered
unexpected overlaps in research priorities across institutes.  For
example, many institutes study angiogenesis, the formation of new
blood vessels; as a treatment for cancer, in heart imaging, the
molecular basis of angiogenesis in the eye, and how angiogenesis might
signal complications in diabetes.

\section{How Fields Change over Time}
\label{sec:sci_change}

One way that science is unique from the fields discussed in the
previous chapters is that it is part of a continuous dialog.  Each
paper in its own way stands on the shoulders of giants. Topic models
for science thus need to be aware of the connections between documents
over time.  Another way that science is different is that the
documents themselves introduce new ideas (we discuss detecting these
innovative ideas in the next section).

One of the first techniques to do this viewed topics as subtly
changing each year with a Dynamic Topic Model~\citep[\abr{dtm}]{blei-06b}.  Within the generative model,
this views each year's topic distribution to be distinct.  That is,
the \underline{physics} topic has separate distributions over words
for each year.  Of course, we don't want the topics to be completely
different every year---we want topics to change, but not \emph{too
  much}.

The \abr{dtm} views topics as changing through \emph{Brownian
  motion}: the topic at year $t$ is drawn from a Gaussian distribution
with mean at the topic for year $t-1$ (a separate variance parameter
controls how much topics can vary each year).  At this point, you may
object given our discussion of distributions from
Chapter~\ref{sec:intro_building_blocks}: Gaussian produce continuous
observations, while topics are multinomial distributions over discrete
outcomes.

To move from Gaussian draws from $\vec x \in \R^d$ to a discrete distributions
over $d$ outcomes, \citet{blei-06b} use the logistic normal form to
create multinomial distribution
\begin{equation}
p(w=k \g \vec x)  = \frac{x_k}{\sum_i x_i}.
\end{equation}
This greatly complicates inference, but allows the topics to change
gradually from year to year.

With this model, the \abr{dtm} discovers how fields change over
time.  At the start of the twentieth century, the language of
\underline{physics} focused on understanding how the ``ether''
propagates waves and the fundamental forces; by midcentury,
understanding ``quantum'' effects took precedence; by the end of the
century, experimental physics with large particle accelerators lead
the search for ever more exotic members of the subatomic menagerie.
While the final topic is nearly unrecognizable given the first, they
all are clearly \underline{physics}; the modeling assumptions of the
\abr{dtm} capture these nearly imperceptible changes in each year.

The flipping of a calendar page does not rule science, however;
changes can happen at any time.  \citet{wang-08} captures changes in
topics in continous time; each document gets its ``own'' view of a
topic that can change slightly from the previous version of a topic.
This can help capture sudden changes in scientific topics, e.g. from
an innovative contribution.

% Author-topic model?

\section{Innovation}

The changes to fields happen because of \emph{innovation}.  Scientists develop
new techniques, new terminologies, and new understanding of the world.  These
concepts require new words which are reflected in their scientific
publications.  Unlike other fields, where documents merely report the changing
world, scientific documents are themselves the force that can change the world:
from Darwin's \textit{Origin of Species} to Einstein's papers on relativity.

Thus, it is interesting to try to find out where this change is happening.  From
a historical perspective, it's interesting to learn who introduced
groundbreaking research first.  For policy makers~\citep{largent-12}, learning
what research collaborations, conditions, and teams lead to breakthroughs can
help direct new initiatives to recreate the magic that lead to these findings.

From a topic perspective, this amounts to detecting \emph{who} was responsible
for changing topics.  From an institutional perspective, \citet{ramage-10} take
a \textit{post hoc} perspective: after fitting a standard \abr{lda} topic model,
find the distribution over research topics in the entire research community at
time $t$ and then look back at time $t-1$ at the places who look like the
future.  They hypothesis is that these places ``lead'' other places to adopt
their ideas.  For example, they found that \dots
% TODO: add example

Capturing more nuanced effects at either the individual or lab level requires
refining the model.  \citet{gerrish-10} adapt the random walk model of
\citet{wang-08} for scientific change.  Instead of topic randomly bumbling into
new concepts, \citet{gerrish-10} posit that innovative article ``nudge'' topics
to look more like them in the future.  This model is called the ``Dynamic
Influence Model'' (\abr{dim}).

For example, the Penn Treebank~\citep{marcus-93} revolutionized natural language
process and helped enabled the statistical revolution in computational
linguistics.  Among its many effects is that people started using the word
``treebank'' much more than they had in the past.  \abr{dim} captures this by
explicitly modeling the influence $\delta_{k,d}$ of a document $d$ in topic
$k$.  This document also has a distribution over words $\tau_d$ (for example, the article
introducing the Penn Treebank uses ``treebank'' much more than ``potato'').
Recall that each topic is a distribution over words at time $t$, $\phi_{t,k}$.

% TODO: make notation consistent with paper and rest of this survey
Documents that don't make a splash have zero influence, while influential
documents are absorbed by other scientists who adopt the influential ideas and
language,
\begin{equation}
  \phi_{t+1,k} \propto \sum_d \delta_{k,d} \tau_d.
\end{equation}
Each of these terms are random variables; inference in the model discovers the
settings of the random variables that best explain the data.  The \abr{dim}'s
estimates of influence correlate well with the number of citations an article
gets (the traditional measure of influence).  Unlike citations, however, the
\abr{dim} can be used in more informal settings to detect influential speech.

While science communication is ostensibly about promoting ideas and new
understanding, topic models can also help understand less objective
communications where influence isn't just about facts but also is about emotion,
beliefs, and relationships.  The next chapter discusses how topic models can
understand these messy, interesting properties of text.