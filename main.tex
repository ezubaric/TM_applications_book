\documentclass[openany]{now} % creates the journal version
% \documentclass{now}  % creates the book pdf version

% the now document class sets various dimensions, so be sure to *not* set
% or alter dimensions in your latex code.
% be sure to remove all manual formatting commands such \newpage, \clearpage.

% a few definitions that are *not* needed in general:
\newcommand{\ie}{\emph{i.e.}}
\newcommand{\eg}{\emph{e.g.}}
\newcommand{\etc}{\emph{etc}}
\newcommand{\now}{\textsc{now}}

\title{Applications of Topic Models}

\author{
Jordan Boyd-Graber \\
University of Colorado \\
\texttt{Jordan.Boyd.Graber@colorado.edu}
\and
Yuening Hu \\
Yahoo! \\
\texttt{ynhu@yahoo-inc.com}
\and
David Mimno \\
Cornell University \\
\texttt{mimno@cornell.edu}
}

\begin{document}

% the following settings can be set or left blank at first
\copyrightowner{}
% \volume{1}
% \issue{3}
% \pubyear{2014}
% \isbn{978-0521833783}
% \doi{1234567890}
% \firstpage{23}
% \lastpage{94}

\frontmatter  % title page, contents, catalog information

\maketitle

\tableofcontents

\mainmatter

\begin{abstract}
This document describes how to prepare a \LaTeX\ document written using the
standard article or book format in the format required for a Foundations and
Trends\textsuperscript{\textregistered}  journal.  You should view the PDF
output (this document, \texttt{sample.pdf}) along with the \LaTeX\ source code
(\texttt{sample.tex}) that created it to get the full picture.
\end{abstract}


\chapter{The What and Wherefore of Topic Models}
\label{ch:intro}

Imagine that you are an intrepid reporter with an amazing scoop: you have
twenty-four hours of exclusive access three decades of e-mails sent within a
corrupt corporation.  You know there's dirt and scandal there, but it has been
well-concealed by the corporation's political friends.  How are you going to
understand this haystack well enough to explain it to your devoted readers under
such a tight deadline?

\section{Tell Me about Your Haystack}

Unlike the vignette above, interacting with large text data sets is often posed
as a needle in a haystack problem.  The poor user---faced with documents that
would take a decade to read---is looking for a single needle: a document (or at
most a handful of documents) that matches what the user is looking for.  This
could be a ``smoking gun'' e-mail, the document that best represents a
concept~\citep{Salton-68} or the answer to a question~\citep{Hirschman-01}.

These questions are important.  The discipline of information retrieval is
built upon systematizing, solving, and evaluating this problem.  Google's empire
is built on the premise of users typing a few keywords into a search engine box
and seeing quick, consistent search results.  However, this is not the only
problem that confronts those interacting with large text datasets.

A different, but related problem is \emph{understanding} large document
collections, common in science policy~\citep{talley-11}, journalism, and the
humanities~\citep{moretti-13}.  There is not just one precious needle in
the haystack.  At the risk of abusing the metaphor, \emph{sometimes you care
  about the straw}.  Instead of looking for a smoking gun alerting to you some
crime that was committed, perhaps you are looking for a sin of
omission: did this company never talk about diversity in its workforce?
Instead of a single answer to a question, perhaps you are looking for a diversity
of responses: what are the different ways that people account for rising income
inequality?  Instead of looking for one document, perhaps you want to provide
population level statistics: what proportion of Twitter users have ever talked
about gun violence?

At first, it might seem that answering these questions would require building an
extensive ontology or categorization scheme.  For every new corpus, you would
need to define  the buckets that a document could fit into, politely ask some librarians and
archivists to put each document into the correct buckets, perhaps automate the
process with some supervised machine learning, and then collect summary
statistics when you are done.

Obviously, such laborious processes are possible---they have been done
for labeling congressional
speeches\footnote{\url{www.congressionalbills.org/}} and understanding
emotional state~\citep{wiebestates}---and remain an important part of
social science, information science, library science, and machine
learning.  But these processes are not always possible, fast, or even
the optimal outcome if we had infinite resources.  First, they
 require a significant investment of time and resources.
Even creating the \emph{list} of categories is a difficult task and
requires careful deliberation and calibration.  Even if it were possible, a
particular question might not warrant the time or effort: the \oe{}uvre
of a minor author (only of interest to a few), or the tweets of a day
(not relevant tomorrow).

This survey explores the ways that humans and
computers make sense of document collections through tools called topic models.
Topic models allow us to answer big-picture questions quickly, cheaply, and without human intervention.
Once trained, they provide a framework for humans to understand document collections both directly by ``reading'' models or indirectly by using topics as input variables for further analysis.
For readers already comfortable with topic models, feel free to skip this
chapter; we will mostly cover the definitions and implementations of topic models.

The intended audience of this book is a reader with some knowledge of
document processing (e.g., knows what ``tokens'' and ``documents''
are), basic understanding of some probability (e.g., what a
distribution is), and interested in many application domains.  We
discuss the information needs of each application area, and how those
specific needs affect models, curation procedures, and
interpretations.

By the end of the book (Chapter~\ref{ch:building}), we hope that
readers will be excited enough to attempt to embark on building their
own topic models.  In this chapter, we go deeper into more of the
implementation details.  Readers who are already topic model experts
will likely not learn much technically, but we hope our coverage of
diverse applications will expose a topic modeling expert to models and
approaches they had not seen before.

\section{What is a Topic Model}

Returning to our motivating example, consider the e-mails from Enron, the prototypical
troubled corporation of the turn of the century.  A source has provided you with a trove of emails, and your editor is demanding an article by yesterday.  You know
that wrongdoing happened, but you do not know who did it or how it was planned
and carried out.  You have suspicions (e.g., around the California energy spot market),
but you are curious to know if there are other skeletons in the closet, and
you are highly motivated to find them.

So you run a topic model on the data.  True to its name, a topic model
gives you ``topics'', collections of words that make sense together.
Looking at the Enron e-mails, we can see topics about gas contracts,
California regulators, and stock prices
(Figure~\ref{fig:enron_topics}).

\begin{figure}
\begin{center}
\rowcolors{2}{gray!25}{white}
\begin{tabular}{cp{10cm}}
\hline
\rowcolor{gray!50}
\hline
Topic & Terms \\
\hline \hline
3 & trading financial trade product price  \\
6 & gas capacity deal pipeline contract \\
9 & state california davis power utilities \\
14 & ferc issue order party case \\
22 & group meeting team process plan \\
\hline
\end{tabular}
\end{center}

  \caption{Five topics from a twenty-five topic model fit on Enron
    e-mails.  Example topics concern financial transactions, natural
    gas, the California utilities, federal regulation, and planning
    meetings.  We provide the five most probable words from each topic
  (each topic is a distribution over all words).}
  \label{fig:enron_topics}
\end{figure}

The first half of a topic model connects topics to a jumbled ``bag of words''.
When we say that a topic is about $X$, we are manually assigning a \textit{post hoc} label
 (more on this in Chapter~\ref{sec:display}).
It remains the responsibility of the human consumer of topic models to go further and
make sense of these piles of straw (we discuss labeling the topics
more in Chapter~\ref{ch:viz}).

\begin{figure}

\begin{center}
\colorbox{lightgray}{ \parbox{.9\linewidth}{
Yesterday, SDG\&E filed a motion for adoption of an
electric procurement cost recovery mechanism and for an order
shortening time for parties to file comments on the mechanism. The
attached email from SDG\&E contains the motion, an executive summary,
and a detailed summary of their proposals and recommendations
governing procurement of the net short energy requirements for
SDG\&E's customers. The utility requests a 15-day comment period, which
means comments would have to be filed by September 10 (September 8
is a Saturday). Reply comments would be filed 10 days later.}}

\begin{tabular}{ccl}
  Topic & Probability \\
  \hline
  9 & 0.42  \\
  11 & 0.05 \\
  8 & 0.05 \\
  \hline
\end{tabular}
\end{center}
  \caption{Example document from the Enron corpus and its association
    to topics.  Although it does not contain the word ``California'',
    it discusses a single California utility's dissatisfaction with how
    much it is paying for electricity.}
  \label{fig:enron_doc}
\end{figure}

Making sense of one of these word piles by itself can be difficult.
The second half of a topic model links topics to individual documents.
For example, the document in
Figure~\ref{fig:enron_doc} is about a California utility's reaction to
the short-term electricity market and exemplifies Topic~9 from
Figure~\ref{fig:enron_topics}.
Considering examples of documents that are strongly connected to a topic, along with the words associated with the topic, can give us a more complete representation of the topic.
If we get a sense that Topic~9 is of
interest, we can explore deeper to find other documents.

\section{Foundations}

\begin{center}
\begin{figure}
  \begin{center}
  \includegraphics[width=.8\linewidth]{figures/matrix_factorization}
  \end{center}

  \caption{A matrix formulation of finding $K$ topics for a dataset
    with $M$ documents and $V$ unique words.  While this view of topic
    modeling includes approaches such as latent semantic analysis
    (\abr{lsa}, where the approximation is based on \abr{svd}), we
    focus on probabilistic techniques in the rest of this survey.}
  \label{fig:matrix_topics}
\end{figure}
\end{center}

You might notice that we are using the general term ``topic model''.
There are many specific mathematical formulations of topic models, and many algorithms that learn the parameters of those models from data.
Although we will focus on particular models and algorithms, we choose our terminology in order to emphasize that the similarities between formulations, models, and algorithms are often greater than their differences.

\index{latent semantic analysis}
Topic modeling began with a linear algebra approach~\citep{deerwester-90} called
latent semantic analysis (\abr{lsa}): find the best low rank approximation of a
document-term matrix (Figure~\ref{fig:matrix_topics}).  While these approaches
have seen a resurgence in recent years~\citep{anandkumar-12,arora-13}, we
focus on probabilistic approaches~\citep{hofmann-99,papadimitriou-00,blei-03},
which are intuitive, work well, and allow for
easy modification (as we see later in many of our later chapters).

\index{latent Dirichlet allocation}
\index{probabilistic latent semantic analysis}
The two foundational probabilistic topic models are latent Dirichlet
allocation~\citep[\abr{lda}]{blei-03} and probabilistic latent
semantic analysis~\citep[\plsa{}]{hofmann-99}.  We describe the
former in significant detail in Chapter~\ref{sec:lda}, but we want to
take a moment to address some of the historical connection between
these two models.

\plsa{} was historically first and laid the foundation for \abr{lda}.
\plsa{} was used extensively in many applications such as information
retrieval.  However, this survey focuses on \abr{lda} because more
researchers have not just \emph{used} \abr{lda}---they have also
\emph{extended} it.  \abr{lda} is not just widely used, but it's also
widely modified.  Because of these prolific modifcations, we feel it
is more important to focus on the mechanics of \abr{lda}, which many
researchers have used as the foundations of new models.  However, as
we explain below (Chapter~\ref{sec:plsa-vs-lda}), the similarities
between \plsa{} and \abr{lda} outweigh the differences.

\begin{figure}
\small
\rowcolors{2}{gray!25}{white}
  \begin{tabular}{cccc}
\hline
\rowcolor{gray!50}
 & & Example & Example \\
\rowcolor{gray!50}
    Distribution & Density & Parameters & Draws \\
    \hline \hline
    Gaussian  & $\frac{1}{\sqrt{2 \sigma^2 \pi}} e^{- \frac{(x-\mu)^2}{2 \sigma^2}}$ & $\mu=2, \sigma^2=1.1$ & $x=2.21$\\
    Discrete  & $\prod_i \phi_i^{\ind{w=i}}$ & $\phi=\begin{bmatrix}
           0.1 \\
           0.6 \\
           0.3 
         \end{bmatrix}$
                                                & $w=2$ \\
   Dirichlet & $\frac{\prod_{i=1}^K \Gamma(\alpha_i)}{\Gamma \left( \sum_{i=1}^K \alpha_i \right)} \prod_{i=1}^K \theta_i^{\alpha_i - 1} $ & $\alpha = \begin{bmatrix}
           1.1 \\
           0.1 \\
           0.1 
         \end{bmatrix}$  & $\theta = \begin{bmatrix}
           0.8 \\
           0.15 \\
           0.05 
         \end{bmatrix}$ \\
     \hline
  \end{tabular}
  \caption{Examples of probability distributions used in the
    generative stories of topic models.  In the case of the discrete
    draw, $w=2$ denotes that the second element (the one with
    probability $0.6$) was drawn.}
  \label{fig:distribution_examples}
\end{figure}

In any technical field it is common for general terms to take on specific, concrete meanings, and this can be a source of confusion.
In topic modeling the word ``topic'' takes on the specific meaning of a probability distribution over words, while still alluding the to more general meaning of a theme or subject of discourse.
Other areas of information retrieval have similarly developed specific meanings for the word ``topic'', and it is important to distinguish them.
There is a substantial and well-developed field of topic detection and tracking~\citep{allan-02}.
In \abr{tdt}, a ``topic'' is usually closer to an event or an individual story.
In contrast, topic models tend to identify more abstract latent factors.
For example, a \abr{tdt} topic might include an earthquake in Haiti, whereas a topic model might represent the same event as a combination of topics such as Haiti, natural disasters, and international aid.

There has been some work on using topic models to detect emerging events by searching for changes in topic probability~\citep{alsumait-08}.
But these methods tend to identify mainly the fact that an event has occurred, without necessarily identifying the specific features of that event.
Other work has found that more lexically specific methods than topic
models are best for identifying memes and viral
phrases~\citep{leskovec-09}. \\


\subsection{Probabilistic Building Blocks}
\label{sec:intro_building_blocks}

In probabilistic models we want to find values for unobserved model variables that do a good job of explaining the observed data.
The first step in inference is to turn this process around, and assert a way to generate data given model variables.
Probabilistic models thus begin with a generative story: a recipe listing a sequence of random events
that creates the dataset we are trying to explain.
Figure~\ref{fig:distribution_examples} lists some of the key players in these
stories, how they are parameterized and what samples drawn from these distributions look like.  We will
briefly discuss them, as we will use them to build a wide variety of topic models later.

\paragraph{Gaussian} If you know any probability distribution already,
it is (probably) the
Gaussian.  This distribution does not have a role in the most basic topic models that we will
discuss here, but it will later (e.g., Chapter~\ref{ch:css}).  We
include it because it is a useful point of comparison against the other
distributions we {\em are} using (since it is perhaps the easiest to understand and best
known). A Gaussian is a distribution over all real numbers (e.g., $0.0, 0.5,
-4.2, \pi$, \dots).  You can ask it to spit out a number, and it will give you
some real number between negative infinity and positive infinity.  But not all
numbers have equal probability.  Gaussian distributions are parameterized by a
mean $\mu$ and variance $\sigma^2$.  Most samples from the distribution will be
near the mean $\mu$; how close is determined by the variance: higher variances
will cause the samples to be more spread out.

\paragraph{Discrete}

While Gaussian distributions are over a continuous space, documents are
combinations of discrete symbols, usually word tokens.\footnote{An emerging trend in natural language
  processing research is to view words as embedded in a continuous space. We
  discuss these ``representation learning'' approaches and their connection to
  topic modeling in Chapter~\ref{ch:conc}, but even then models are still defined over a discrete set of words.}   Thus, we need a distribution
over discrete sets.

A useful metaphor for thinking about discrete distributions is a weighted die.
The number of faces on the die is its dimension, and each face is associated with a distinct
outcome.  Each face has its own probability of how likely that outcome is;
these probabilities are the parameters of a discrete distribution
(Figure~\ref{fig:distribution_examples}).

Topic models are described by discrete distributions (sometimes called
multinomial distributions) that describe the connection between words and topics (the first half) and topics and documents (the second half).  A distribution
over words is called a topic distribution; each of the topics gives
higher weights to some words more than others (e.g., in Topic 9 from
the Enron corpus, ``state'' and ``california'' have higher probability
than other words).  Each document also has an ``allocation'' for each
topic: documents are about a small handful of topics, and most
documents have very low weights for most of the possible topics.

\paragraph{Dirichlet}

Although discrete distributions are the star players in topic models, they are
not the end of the story.  We often begin with Dirichlet distributions.
Just as Gaussians produce real numbers and discrete distributions produce symbols from a finite set, Dirichlet distributions produce probability vectors that can be used as the parameters of discrete distributions.
Like the Gaussian distribution, they have parameters analogous to a mean and
variance.  The mean is called the ``base measure'' $\tau$ and is the expected
value of the Dirichlet distribution: the values you would get if you averaged many draws from the
Dirichlet.  The concentration parameter $\alpha_0$ controls how far away individual draws are from the base measure.
Note that we often combine these parameters into a single value for each dimension: $\alpha_k = \alpha_0 \tau_k$.

\begin{center}
\begin{figure}
  \centering
  \includegraphics[width=.8\linewidth]{figures/dirichlet}
  \caption{Given different Dirichlet parameters, the Dirichlet
    distribution can either be informative (left, middle) or sparse
    (right).  Sparse distributions encourage distributions to favor
    only a small number of elements but do not care which ones.  This
    is consistent with our intuitions of how documents are written:
    they are only about a few things, and topics contain only a
    handful of words.}
  \label{fig:dirichlet_sparsity}
\end{figure}
\end{center}


If $\alpha_0$ is very large, then the draws from a Dirichlet will be very close to
$\tau$ (Figure~\ref{fig:dirichlet_sparsity}, left).  If $\alpha_0$ is small, however,
something more interesting happens: the discrete distributions become sparse
(Figure~\ref{fig:dirichlet_sparsity}, right).  A sparse distribution is a
distribution where only a few values have high probability and all other values are small.

Because topic models are meant to reflect the properties of real documents, modeling sparsity is important.  When a person sits down to write a document, they only
write about a handful of the topics that they could potentially use.  They do not write about every possible topic, and the sparsity of Dirichlet distributions is the probabilistic
tool that encodes this intuition.

There are several important special cases of the Dirichlet distribution.
If the base measure $\tau$ is uniform, we call the resulting distribution {\em symmetric}.
This case is appropriate when we do not expect any one element to be, on average, more likely than any other element across all samples from the distribution.
In the symmetric case the  distribution has only one parameter, the concentration $\alpha_0$.
If the base measure is uniform and the concentration parameter $\alpha_0$ is equal to the number of dimensions $K$ (or, equivalently, $\alpha_k = 1.0$ for all $k$), the distribution is uniform, placing equal probability on all $K$-dimensional probability distributions.


\section{Latent Dirichlet Allocation}
\label{sec:lda}

We now have all the tools we need to tell the complete story of the most popular topic model: latent Dirichlet allocation~\citep{blei-03}.  Latent
Dirichlet allocation\footnote{The name \abr{lda} is a play on \abr{lsa}, its
  non-probabilistic forerunner (latent semantic analysis).  Latent because we
  use probabilistic inference to infer missing probabilistic pieces of the
  generative story.  Dirichlet because of the Dirichlet parameters encoding
  sparsity.  Allocation because the Dirichlet distribution encodes the prior for
  each document's allocation over topics.} posits a ``generative process'' about
how the data came to be.  We assemble the probabilistic pieces to tell
this story about generating topics and how those topics are used to create
diverse documents.

\paragraph{Generating Topics}

The first part of the story is to create the topics.  The user specifies that
there are $K$ distinct topics.  Each of the $K$ topics is drawn from a Dirichlet
distribution with a uniform base distribution and concentration parameter
$\lambda$: $\phi_k \sim \dir{\lambda {\bm u}}$.  The discrete distribution
$\phi_k$ has a weight for \emph{every} word in the vocabulary.

However, when we summarize topics (as in
Figure~\ref{fig:enron_topics}), we typically only use the top (most probable) words of
a topic.  The lower probability words are less relevant to the topic
and thus are not shown.

\paragraph{Document Allocations}

Document allocations are distributions over topics for each document.  This
encodes what a document is about; the sparsity of the Dirichlet distribution's
concentration parameter $\alpha_0$ ensures that the document will only be about a
few topics.  Each document has a discrete distribution over topic: $\theta_d \sim
\dir{\alpha {\bm u}}$.

\paragraph{Words in Context}

Now that we know what each document is about, we need to actually create the
words that appear in the document.  We assume\footnote{It is possible to model
  this in the generative story as well, e.g., with a Poisson distribution.
  However, we often do not care about document \emph{lengths}---only what the
  document is about---so we can usually ignore this part of the story.} that
there are $N_d$ words in document $d$.  For each word $n$ in the
document $d$, we first choose a {\bf topic assignment} $z_{d,n} \sim
\disc{\theta_d}$.  This is one of the $K$ topics that tells us which topic the
word token is from, but not what the word is.

To select which word we will see in the document, we draw from a discrete
distribution again.  Given a word token's topic assignment $z_{d,n}$, we draw from that
topic to select the word: $w_{d,n} \sim \phi_{z_{d,n}}$.  The topic assignment
tells you what the word is about, and then this selects which distribution over
words we use to generate the word.  % (Figure~\ref{fig:generative-ex}).

For example, consider the document in Figure~\ref{fig:enron_doc}.  To
generate it, we choose a distribution over all of the topics.  This is
$\theta$.  For this document, the distribution favors Topic~9 about
\underline{California}.  The value for this topic is higher than any other topic.  For
each word in the document, the generative process chooses a topic
assignment $z_n$.  For this document, any topic is theoretically possible, but we expect that most of those will be Topic~9.

Then, for each token in the document, we need to choose which word type will appear.  This
comes from Topic~9's distribution over words (multiple topics have
word distributions shown in Figure~\ref{fig:enron_topics}).  Each is a
discrete draw from the topic's word distribution, which makes words
like ``California'', ``state'', and ``Sacramento'' more likely.

It goes without saying that the generative story is a fiction~\citep{box-87}.
Nobody is sitting down with dice to decide what to type in on their keyboard.
We use this story because it is \emph{useful}.  This fanciful story about randomly
choosing a topic for each word can help us because if we assume this generative
process, we can work backwards to find the topics that explain how a document
collection was created: every word, every document, gets associated with these
underlying topics.

This simple model helps us order our document collection: by assuming this story, we
can discover \emph{topics} (which certainly do not exist) so we can understand
the common themes that people use to write documents.  As we will see in later
chapters, slight tweaks of this generative story allow us to uncover more
complicated structures: how authors prefer specific topics, how topics change
over time, or how topics can be used across languages.

\section{Inference}

Given a generative model and some data, the process of uncovering the hidden
 pieces of the probabilistic generative story is called \emph{inference}.  More
concretely, it is a recipe for generating algorithms to go from data to
\emph{topics that explain a dataset}.

There are many flavors of algorithms for posterior inference: message
passing~\citep{zeng-13}, variational inference~\citep{blei-03},
gradient descent~\citep{hoffman-10}, and Gibbs
sampling~\citep{griffiths-04}.  All of these algorithms have their
advocates and reasons you should use them.  In this survey, we focus
on Gibbs sampling, which is simple, intuitive, and---with some clever
tricks specific to topic models---fast~\citep{yao-09}.  (We discuss
variational inference in Chapter~\ref{ch:building}.)

We present the results of Gibbs sampling without derivation,
which---along with the history of its origin in statistical
physics---are well described elsewhere.\footnote{We recommend
  \citet{resnik-09} for additional information on derivation.} We use
a variety of Gibbs sampling called \emph{collapsed} Gibbs sampling,
which allows inference to side-step some of the pieces of the
generative story: instead of explicitly representing the parameters of
a discrete distribution, distinct from any observations drawn from
that distribution, we represent the distribution solely through those
observations.  We can then recreate the topic and document
distributions through simple formulas as they are needed.

\subsection{Random Variables}

\paragraph{Topic Assignments}

Since every individual token is assumed to be generated from a single topic,
we can consider the {\em topic assignment} of a token as a variable.  For example,
an instance of the word ``compilation'' might be in a \underline{computer} topic in one
document and in an \underline{arts} topic in another document.  Because each token has its own
topic assignment, it is even possible that the
same word might be assigned to different topics in the same document.
To estimate \emph{global} properties of the topic model we use aggregate statistics derived from these token-level topic assignments.

\paragraph{Document Allocation} The document allocation is a distribution over
the topics for each document; in other words, it says how popular each topic is
in a document.  If we count up how often a document uses a topic, this gives us
an idea of the popularity.  Let's define $\dc{d}{i}$ as the number of times
document~$d$ uses topic~$i$.  Clearly, this is larger for more popular topics;
however, it is not a probability because it is larger than one.  We can make it a
probability by dividing by the number of words in a document
\begin{equation}
\frac{\dc{d}{i}}{\sum_k \dc{d}{k}},
\label{eq:theta_ml}
\end{equation}
but this is problematic because it can sometimes give us zero and ignores the
influence of the Dirichlet distribution; a better estimate is\footnote{To be
  technical, Equation~\ref{eq:theta_ml} is a maximum likelihood estimate and
  Equation~\ref{eq:theta_map} is the maximum \textit{a posteriori}, which
  incorporates the influence of both the prior and the data.}
\begin{equation}
\theta_{d,i} \approx \frac{\dc{d}{i} + \alpha_i}{\sum_k \dc{d}{k} + \alpha_k}.
\label{eq:theta_map}
\end{equation}
It is important that this is never zero because we do not want it to rule out the possibility
that a topic is used in a particular document.  This helps the sampler
explore more of the possible combinations.

\paragraph{Topics}

Each topic is a distribution over words.  To understand what a topic is about,
we look at the profile of all of the tokens that have been assigned to that
topic.  We estimate the probability of a word in a topic as
\begin{equation}
\phi_{i,v} \approx \frac{\tc{i}{v} + \beta_v}{\sum_w \tc{i}{w} + \beta_w},
\label{eq:phi_map}
\end{equation}
where $\beta$ is the Dirichlet parameter for the topic distribution.

\subsection{Algorithm}

The collapsed Gibbs sampling algorithm for learning a topic model is
only based on the topic assignments, but we will use our estimates for
the topics $\phi_k$ and the documents $\theta_d$ discussed above.  We
begin by setting topic assignments randomly: if we have $K$ topics,
each word has equal chance to be associated with any of the topics.
These topics will be quite bad, looking like noisy copies of the
overall corpus distribution. But we will improve them one word at a
time.

The algorithm proceeds by sweeping over all word tokens in turn over and over.
At each iteration we change the topic assignments for each word in a way the reflects the
underlying probabilistic model of the data.  On average, each pass over the data makes the
topics slightly better until the model reaches a steady state.  There is no easy way
to tell when such a steady state has been reached, but eventually the topics will ``converge'' to something reasonable and you can consider yourself done.

The equation for the probability of assigning a word to a particular topic
combines information about words and about documents\footnote{To be theoretically correct, it is important
not to include the count associated with the token you are currently sampling in
these counts, which becomes more clear if the probability is written as
$p(z_{d,n}=j\,|\,z_{d,1}\dots z_{d,n-1},z_{d,n+1}\dots z_{d,N_d}, w_{d,n})$ to
show the dependence on the topic assignments of \emph{all other} tokens but not
this token.}
\begin{align}
p(z_{d,n}=i \g \dots) = \theta_d
\phi_ji= \left(\frac{\dc{d}{i} + \alpha_i}{\sum_k \dc{d}{k} + \alpha_k} \right) \left( \frac{\tc{i}{w_{d,n}} + \beta_v}{\sum_w \tc{i}{w} +
    \beta_w} \right).
\label{eq:sampling}
\end{align}
Computing this value for each topic will result in a probability distribution over the topic assignment for this word token, given all the other topic assignments.  The next step is to randomly choose one of those indices with
probability proportional to the vector value.  You now assign that word to the
topic, update $\dc{d}{\cdot}$ and $\tc{\cdot}{w_{d,n}}$, and move on to the next word and repeat.
The two terms provide two ``pressures'', for global and local coherence. Sparsity in the topic-word distributions encourages tokens of the same word type to be assigned to a small number of topics,  regardless of where they occur. Sparsity in the document-topic distributions encourages tokens in the same document to be assigned to a small number of topics, regardless of what type they are.
For example, knowing that a word is ``compilation'' narrows down the
number of potential topics considerably, but leaves ambiguity: is it
\underline{computer} compilation or a \underline{music} compilation? Knowing that the word occurs in a document with many other words in the \emph{arts} topic resolves this ambiguity, leaving the \emph{arts} topic as the most probable assignment.

At the very end of the algorithm, we can use the estimates of each topic
(Equation~\ref{eq:phi_map}) to summarize the main themes of the corpus and the
estimates of each document's topic distribution (Equation~\ref{eq:theta_map}) to
start exploring the collection automatically (Chapter~\ref{ch:ir}) or with a
human in the loop (Chapter~\ref{ch:viz}).

The algorithm that we have sketched here is the foundation of many of the more
advanced models that we will discuss later in the survey.  While we will not describe
the algorithms in detail, we will occasionally make reference to this sketch to
highlight challenges or difficulties in implementing topic models.

\subsection{Plate Diagrams}
\label{sec:plate}

Plate diagrams provide a shorthand for quickly explaining which random
variables are associated with each other.  If you look up many of the
references used in this survey, you'll likely see plate diagrams (we
also use a plate diagram later in Figure~\ref{fig:ptm-simplified}).  

\begin{figure}
  \begin{center}
  \includegraphics[width=0.7\linewidth]{figures/lda_plate}
  \end{center}
  \caption{Plate diagram for \abr{lda}.  Nodes show random variables,
    lines show (possible) probabilistic dependence, rectangles show
    repetition, and shading shows observation.}
  \label{fig:plate-lda}
\end{figure}

Let's begin with a plate diagram for \abr{lda}
(Figure~\ref{fig:plate-lda}).  You can compare these to the generative
story in Chapter~\ref{sec:lda}.  All of the random variables are
there, each in its own circle.  The lines between random variables
tell more of the story.  You can see that if a random variable is
conditioned on another, there is a line going from the variable that
is \emph{conditioned on} {\bf to} the variable that is
\emph{conditionally dependent}.  For example, a word depends on the
token assignment $z_{d,n}$ and a topic $\phi_k$, so there are lines
from both.

You can think about the rectangular boxes as repetition.  The letter
in the bottom right of the box shows how often what's inside the box
is replicated.  There is a box for each document (there are $M$ in
total) and each token (the box of words is inside the box for
documents, so there are many more tokens than documents).

When a variable is shaded, this means that it is observed.  These are
the data we start with.  The unshaded variables must either be
inferred (e.g., topics $\phi$) or are hyperparameters that must be set
or inferred (e.g., Dirichlet parameter $\alpha$).

Plate diagrams allow a reader to quickly see a ``family resemblance''
between related models, and once someone has become fully immersed in
topic models, it's often possible to at a glance understand a model
from its plate diagram.  However, plate diagrams are imperfect; they
lack some of the key information you need to understand the model.
For instance, the exact probabilistic relationship between variables
is underspecified.

\subsection{What's so Great about Dirichlet?}
\label{sec:plsa-vs-lda}

Now that we have described what \abr{lda} is, we can return to its
history.  What is the innovation that separates \abr{lda} from
\plsa{}, its predecessor?  Na\"ively, the difference is changing an
``s'' to a ``d'' (i.e., changing p\abr{l{\bf S}a} to \abr{l{\bf
    D}a}).  The deeper story is about as consequential.

Instead of having a Dirichlet prior over $\theta$, \plsa{} assumes
that $\theta$ is simply a multinomial parameter.  In practice, this
means that documents aren't encouraged to focus on a limited number of
topics and often ``spread out'' to have small weights for many
different topics.  In theory, this means that there isn't as sound a
generative story for how a document came to be: you can't run the
generative process forward from scratch if you must have $\theta$ as a
parameter to start with.

These differences are relatively minor.  \abr{lda} has slightly easier
inference---particularly when it comes to tweaking the model---which
has caused it to become the more popular of the two models.  Thus, we
will focus on comparing models to \abr{lda}.  This is
not to diminish from \plsa{} and its incontrovertible place in the
literature, it helps us present a more unified narrative for our reader.

\subsection{Implementations}

Hopefully the previous algorithm sketch has convinced you that implementing
topic models is not a Herculean task; most skilled programmers can complete a
reasonable implementation of topic models in less than a day.  However, we would
suggest not trying to implement basic \abr{lda} if you just want the
output of a topic model, as there are many solid
implementations that help users get to useful results more quickly, particularly
as topic models often require extensive preprocessing.

Mallet is fast and is a widely used implementation in Java~\citep{mallet}.  This
is where you should probably start, in our biased opinion.  It runs in Java, uses
highly-optimized Gibbs sampling implementations, and can work from a variety of
text inputs.  It is well documented, mature, and runs well on a multi-core
machine, allowing it to process up to millions of documents.  Variational
inference is the other major option~\citep{blei-03,vw}, but often requires a
little more effort for new users to get a first result.

However, not all users are comfortable with Java; there are many
implementations available on other platforms and in many programming
languages.\footnote{There are so many and they change so quickly that
  we're reluctant endorse specific ones here.}  Many of these
implementations are well-built, but it may be worthwhile checking
whether they have all of the features of mature implementations like
Mallet so that you know what (if anything) you're missing.

% Smola, Amr
However, if your corpus is truly large, it may be worthwhile
considering techniques that can be parallelized over large computer
clusters.  These techniques can be based on variational
inference~\citep{Narayanamurthy-11,zhai-12} or on
sampling~\citep{newman-08}.

While these implementations allow you to run \emph{specific} topic
models, other frameworks allow you to specify arbitrary generative
models.  This allows for quick prototyping of topic models and
integrating topic models with other probabilistic frameworks like
regression or collaborative filtering.  Examples of these general
frameworks include Stan~\citep{stan-software:2014},
Theano~\citep{theano}, and Infer.net~\citep{InferNET14}.

If you cannot find the specific model that you want among these
existing software packages, the flexibility and simplicity of topic
models and inference makes it relatively simple to adapt topic models
to model specific phenomena (as we describe in following chapters).

\section{The Rest of this Survey}

In each of the following chapters, we focus on an application of topic
models, gradually increasing the complexity of the underlying models.
The chapters do occasionally refer to each other, but a reader should
be able to read each of the chapters independently.

The next chapter returns to the distinction between high level
overviews and finding a needle in a haystack.  We show how a high
level overview can help users and algorithms find documents of
interest.  We show how a high level overview can help
algorithms (Chapter~\ref{ch:ir}) and users (Chapter~\ref{ch:viz}) find documents
of interest.

These tools help enable new applications of topic models: how
understanding newspapers (Chapter~\ref{ch:nonfiction}) reveals the
march of history, how the corpus of writers of fiction
(Chapter~\ref{ch:fiction}) illuminates societal norms, how the
writings of science reveal innovation (Chapter~\ref{ch:sci}), or
how politicians' speeches (Chapter~\ref{ch:css}) reveal schisms in
political organizations.

Finally, the survey closes with thoughts about how interested
researchers can start building their own topic models
(Chapter~\ref{ch:building}) and how topic models may change in the
future (Chapter~\ref{ch:conc}).

\chapter{Information Retrieval}
\label{ch:ir}

Information Retrieval (IR) systems aim to retrieve relevant documents by comparing query and document texts. Users start with their information need in the form of queries. Early IR systems treat both the query and documents as ``bag of words'', retrieve and rank the documents by measuring the word overlap between queries and documents. 

However, the ability of this direct and simple matching is always limited. Words with similar meaning or in different forms should also be considered as matched instead of being ignored. Also, humans would also use background knowledge to interpret and understand the queries and ``add'' missing words~\citep{wei-07}, which provides another way often referred as \textbf{query expansion} to improve the retrieval and ranking results. 

Both directions can be pursued by learning and discovering the semantic relations between words and further the semantic relations between queries and documents. Topic models, which describe each topic using weighted words and model each document as a distribution over all topics, is an effective way to capture such semantic relations~\citep{deerwester-90,hofmann-99a}.

In this chapter, we briefly introduce the information retrieval framework and two major directions of applying topic models to improve the information retrieval results: query expansion~\citep{Park-2009,Andrzejewski-2011} and probabilistic language modeling \citep{Lu-2011,wei-06}.

%\section{Probabilistic Information Retrieval}

%Latent Semantic Analysis (LSA) proposed by \citet{deerwester-90} makes use of dimensionality reduction to capture the semantic relations among words and documents.  (Briefly review, note distinction with probabilistic approaches.)

%It has been successfully applied into the information retrieval systems through automatic indexing with LSA (Latent Semantic Indexing, LSI) \citep{deerwester-90,dumais-95}.

%The probabilistic Latent Semantic Indexing (pLSI), introduced by \citet{hofmann-99a}, is an aspect model which associates an unobserved class (topic) variable with each observed word occurrence. pLSI provides a better fit to text data than LSI, thus quickly gained acceptance in IR systems.

\section{Language Modeling in IR}

The language modeling approach~\citep{croft-03,PonteCroft,song-99} is one of the main frameworks for using topic models in IR systems, since it has been shown to be effective probabilistic framework for studying information retrieval problem~\citep{PonteCroft,berger-99}.

A statistical language model is to estimate the probability of word sequences, denoted as $p(w_1,w_2,\cdots,w_n)$. In practice, the statistical language model is often approximated by N-gram models. A unigram model assumes each word in the sequence is independent, and is denoted as,
\begin{align}
p(w_1,w_2,\cdots,w_n) = p(w_1)p(w_2) \cdots p(w_n)
\end{align}
A trigram model assumes the probability of the current word only depends on the previous two words, and it is represented as,
\begin{align}
p(w_1,w_2,\cdots,w_n)=p(w_1)p(w_2|w_1)p(w_3|w_1,w_2)\cdots p(w_n|w_{n-2},w_{n-1})
\end{align}

In the application of information retrieval, the queries are generated by a probabilistic language model based on a document~\cite{zhai-01}. More specifically, each document is viewed as a language sample, and a language model for each document is estimated based on document terms. Then the query is generated by each document language model. The probability of a query is computed by multiplying the probabilities of generating each query term using different document language model, and the documents are ranked based on the probability. Higher probability implies the corresponding document is more relevant to the given query~\cite{song-99}.

Given a document sample $d$, a straightforward way to estimate the probability of a term $t$ is to use the maximum likelihood estimation as below,
\begin{align}
p(t|d) = \frac{n_{d,t}}{n_{d,\cdot}}
\end{align}
where $n_{d,t}$ is the term frequency of term $t$ in document $d$, and $n_{d,\cdot}$ is the total number of tokens in document $d$. Then the probability of the given query can be computed. However, a document is often too small to cover all the terms in the query, and the probability of a missing term is zero, which means the probability of the whole query is zero and causes problems for ranking documents. 




The basic language modeling framework is to compute the model likelihood of documents for generating the queries. Topic models, which represent documents with topics, offer a new and interesting means to model documents~\citep{wei-07}. A probability mixture model and a term model with back-off smoothing are presented to integrate topic models in this section.

A term model with back-off smoothing~\citep{katz-87} is another popular framework to use topic models in IR systems, where a term model is learned for each term in a document and the back-off smoothing~\citep{katz-87} is applied.

\section{Query Expansion in IR}

A probability mixture model combines multiple different probability distributions to integrate different factors in IR for query representation or document representation~\citep{miller-99,zhai-01,liu-04}.


\section{Summary}
% Perhaps it would be nice to mention e-discovery?

Transition to next chapter: what if you care about recall and understanding


%%%%%%%%%%%%%%%%%%%%%%%%%%%%%%%%%%%%%%%%%%%%%%%%%%%%%%%%%%%%%%%%%%%%%%%%%%%%%%%%%%%%%%%%%%%%%%%%%%%%%%%%%%%%%%%%%%%%%%%%%
%%%%%%%%%%%%%%%%%%%%%%%%%%%%%%%%%%%%%%%%%%%%%%%%%%%%%%%%%%%%%%%%%%%%%%%%%%%%%%%%%%%%%%%%%%%%%%%%%%%%%%%%%%%%%%%%%%%%%%%%%
%%%%%%%%%%%%%%%%%%%%%%%%%%%%%%%%%%%%%%%%%%%%%%%%%%%%%%%%%%%%%%%%%%%%%%%%%%%%%%%%%%%%%%%%%%%%%%%%%%%%%%%%%%%%%%%%%%%%%%%%%
%%%%%%%%%%%%%%%%%%%%%%%%%%%%%%%%%%%%%%%%%%%%%%%%%%%%%%%%%%%%%%%%%%%%%%%%%%%%%%%%%%%%%%%%%%%%%%%%%%%%%%%%%%%%%%%%%%%%%%%%%
%%%%%%%%%%%%%%%%%%%%%%%%%%%%%%%%%%%%%%%%%%%%%%%%%%%%%%%%%%%%%%%%%%%%%%%%%%%%%%%%%%%%%%%%%%%%%%%%%%%%%%%%%%%%%%%%%%%%%%%%%


\begin{comment}

\chapter{Information Retrieval Structure}
\label{ch:ir_structure}

Topic models, such as Latent Semantic Analysis (LSA) by \citep{deerwester-90} and probabilistic Latent Semantic Indexing~\cite{hofmann-99a}, can capture the semantic relations between documents and queries~\citep{wei-07}.

Topic models have been applied into the information retrieval framework to improve ranking results. Probabilistic language modeling~\citep{croft-03} is a common formalism that incorporates information from topic models.

In this chapter, we briefly introduce the information retrieval framework and the semantic relations between queries and documents. We also present the latent semantic analysis and probabilistic latent semantic indexing models, and how they are combined into information retrieval framework through language modeling.

\section{Semantic Relations in Information Retrieval}

Information Retrieval (IR) systems aim to retrieve relevant documents by comparing query and document texts.

From computer's view, Documents are usually treated as ``bags of words'', and documents are retrieved and ranked by measuring the word overlap. This is consistent with topic models.

Topic models use those bags of words to build abstractions (topics), however.

However, humans would use background knowledge to interpret and understand the queries and ``add'' missing words~\citep{wei-07}, akin to query expansion.

To more accurately retrieve related documents, the semantic relations between queries and documents are needed to improve the ranking results. Topic models, which describe each topic using weighted words and model each document as a distribution over all topics, is an effective way to capture such semantic relations~\citep{deerwester-90,hofmann-99a}.

\section{Topic Models in IR}

Latent Semantic Analysis (LSA) proposed by \citet{deerwester-90} makes use of dimensionality reduction to capture the semantic relations among words and documents.  (Briefly review, note distinction with probabilistic approaches.)

It has been successfully applied into the information retrieval systems through automatic indexing with LSA (Latent Semantic Indexing, LSI) \citep{deerwester-90,dumais-95}.

The probabilistic Latent Semantic Indexing (pLSI), introduced by \citet{hofmann-99a}, is an aspect model which associates an unobserved class (topic) variable with each observed word occurrence. pLSI provides a better fit to text data than LSI, thus quickly gained acceptance in IR systems.

\section{Applying Topic Models into IR}

The language modeling approach~\citep{croft-03,PonteCroft,song-99} is the main framework for using topic models in IR systems, since it has been shown to be effective probabilistic framework for studying information retrieval problem~\citep{PonteCroft,berger-99}.

The basic language modeling framework is to compute the model likelihood of documents for generating the queries. Topic models, which represent documents with topics, offer a new and interesting means to model documents~\citep{wei-07}. A probability mixture model and a term model with back-off smoothing are presented to integrate topic models in this section.

A probability mixture model combines multiple different probability distributions to integrate different factors in IR for query representation or document representation~\citep{miller-99,zhai-01,liu-04}.

A term model with back-off smoothing~\citep{katz-87} is another popular framework to use topic models in IR systems, where a term model is learned for each term in a document and the back-off smoothing~\citep{katz-87} is applied.

% Perhaps it would be nice to mention e-discovery?

Transition to next chapter: what if you care about recall and understanding

\end{comment}
\chapter{Visualization}
\label{ch:viz}

While the previous two chapters have focused on algorithmic uses of topic
models, one of the reasons for using topic models is that they produce
human-readable summaries of the themes of large document collections.  However,
for users to use the results of topic models, they must be able to understand
the models' output.  This depends on \emph{visualization} and \emph{interaction}
with the model.

We begin this chapter with a discussion of how best to show individual topics to
users.  From these foundations, we move to how we can display entire
models---with many topics---to users.  Finally, we close with how users can
provide feedback through these interfaces to improve the underlying model.

\section{Displaying Topics}
\label{sec:display}

Recall from the previous chapter that topics are distributions over words; the
words with the highest weight in a topic best explain what the topic is about.
While the simplest answer---just show the most probable words---is a common
solution, there are possible refinements that can improve a user's understanding
of a dataset by showing the relationships between words or explicitly showing
words' probability.

% cite TACL paper
\paragraph{Word Lists}

Just showing a list of the most common words (a
visualization that we'll call ``word list'') is very simple, it also works well.
Users can quickly understand what's going on, and it is an efficient use of
space.  represented horizontally~\cite{gardner2010topic,smith2015visual} or
vertically~\cite{eisenstein2012topicviz,chaney2012visualizing}, with or without
commas separating the individual words, or using set
notation~\cite{chaney2012visualizing}.  \citet{smith2015visual} go further by
adding bars representing the probabilities of the word.

\paragraph{Word Clouds}

Word clouds (e.g., Figure~\ref{fig:word-cloud}) are another popular approach for
displaying topics.  Unlike word lists, they also use the size of words to convey
additional information. Word clouds typically use the size of words to reflect
the probability of the words.  This uses more of a given visualization area to
be used to display a topic.

However, word clouds have been criticized for providing poor support for visual
search~\cite{Viegas2008} and lacking contextual information between
words~\cite{harris11}; users can sometimes draw false connections between words
that are placed next to each other randomly in a word cloud.  Another
alternative is to use word associations to layout
words~\citep{Smith:Chuang:Hu:Boyd-Graber:Findlater-2014};
Figure~\ref{fig:topic-in-a-box} shows places words that appear together next to
each other in the visualization.

\subsection{When Words aren't Enough}

Multi-word expressions can be discovered through
pre-processing~\citep{talley-11}, post-processing step~\citep{blei-09b},
or a joint model~\citep{johnson-10}.

\section{Labeling Topics}

Throughout this survey, we've been referring to topics about
\underline{information technology} or about \underline{the arts}.  These are
convenient labels, but completely removed from the raw distribution over words.
Thus, it's often useful to assign labels to topics within an interface.

% http://lms.comp.nus.edu.sg/sites/default/files/publication-attachments/pp0406-mao.pdf
% http://www.aclweb.org/anthology/P11-1154
% http://sifaka.cs.uiuc.edu/czhai/pub/kdd07-label.pdf
% http://demeter.inf.ed.ac.uk/redites/docs/vCoreTopicLabel.pdf
% http://www0.cs.ucl.ac.uk/staff/N.Aletras/resources/ACL14_Topic_Labelling.pdf
% http://www.aclweb.org/anthology/C10-2069

Automatic topic labeling \dots

\section{Displaying Models}

But a topic model is more than about individual topics.

It is important to show the most relevant documents for reach
topics~\citep{chaney-12}

More sophisticated techniques can give the relationship between meta
data and topics~\citep{gardner-10,eistenstein-14}

It is also important to show the similarity between
topics~\citep{chuang-12}

Showing the relationships between multiple models can also help
distinguish stable from spurious topics~\citep{chuang-15}

\section{Interaction}

\subsection{Improving Topics}

\subsection{Labeling Topics}
\label{sec:viz:label}

But not all topic models are perfect.  Visualizations can help show
users where topic models have issues.

One approach is to provide probabilistic constraints~\citep{hu-14:itm}

Another approach is to add matrix constraints~\citep{choo-13}

These interactions and visualizations allow users to discover and
refine insights, allowing them to explore and understand diverse
datasets.

\chapter{Historical Documents}
\label{ch:nonfiction}

Topic models play an important role in the analysis of historical documents.
Historical records tend to be extensive and difficult to manage without intense and time-consuming organization.
Records are complicated: they resist categorization, and may even lack standard spelling and formatting.
But there is more to history than simply the management of documents.
The task of a historian is not only to absorb the contents of historical records, but to generalize; to find patterns and regularities that are true to the documents, but also beyond any single piece of evidence.
Topic models are useful because they address these issues. They are scalable, robust to variability, and able to generalize while remaining grounded in observation.

Automated methods are an especially valuable counterpoint to traditional close reading methods.
Studying history is about encountering the unexpected, often in contexts that seem familiar.
We don't necessarily know how people in the past talked about particular issues, or how they organized their lives.
Perhaps more dangerously, we assume that we know these things, and that our ancestors saw the world in the same way we do.
Topic models give us a perspective that is interpretable but at the same time alien, based on patterns in documents and not on our own conceptions of how things should be.

Time is a critical variable in the study of historical documents.
Although many modern collections have a significant aspect of time variation (see for example scientometrics), time is a defining element of historical research.
Collections of historical documents are necessarily situated in a time other than our own, but also tend to cover long periods --- decades or even centuries.
As a result, many of the examples cited in this chapter organize documents along a temporal axis.
The associated analysis is particularly concerned with how language, as reflected in topic concentrations and topic contents, changes over time.

This chapter is organized around different formats for historical documents.
A recurring focus is the desire to plot events and discourses against time.
We begin with historical newspapers, which are relatively close to the modern news articles that are a more familiar use case in topic modeling.
We then consider other forms of historical records, such as annals and diaries.
These demonstrate the flexibility of topic modeling, including a corpus not in English and corpus in English with irregular spelling.
Finally, we consider studies of historical scholarly literature.

%A useful resource:
%Clay Templeton, Topic Modeling in the Humanities: An Overview.\footnote{http://mith.umd.edu/topic-modeling-in-the-humanities-an-overview/}

\section{Newspapers}

\citet{newman-06} present an example of topic modeling on historical newspapers, in a collection of articles from the {\em Pennsylvania Gazette} from 1728 to 1800.\footnote{http://www.accessible-archives.com/}
These articles comprise 25 million word tokens in articles and advertisements, and cover several generations of everyday life before, during, and after the founding of the United States of America.
The authors contrast their study to manually created keyword-based indexes, which focus on specific terms and can be applied inconsistently across large corpora.
Spurious patterns in index term use could complicate historical research.
They cite an example of the tag {\em adv}, which is used extensively in the early and late decades of the corpus, but not in the middle.
The topic-based approach is attractive because it is consistent across the collection (as long as the terms used in the documents are themselves consistent) and because it operates at a more abstract semantic level, reducing the chance that modern historians miss key terms.

They compare three methods for finding semantic dimensions, latent semantic analysis \citep{deerwester-90}, k-means clustering, and a topic model \citep{hofmann-99}.
The difference between these methods can be described in terms of expressivity.
LSA is effective at embedding word types and documents in a low-dimensional space, but the individual dimensions of this space are not interpretable as themes.
LSA is too expressive: it places no constraints, such as positivity, on the learned dimensions, and therefore produces uninterpretable results that nevertheless fit the document set well.
The k-means clustering is more similar to the topic model, and more successful at finding recognizable themes.
But it is also prone to repeating similar clusters with small variations.
Because of the single-membership assumption (a document can only belong to one cluster), the clustering model cannot represent documents with varying combinations of somewhat independent themes.
The k-means model is therefore insufficiently expressive: it it forced to ``waste'' clusters on frequent combinations of simpler themes.
The topic model, in contrast, has both modeling flexibility along with sufficient  constraints to support interpretable results.

The authors find that the learned topics are a good representation of dynamics in the corpus, although not always in a direct manner.
There is a large increase in discussions of politics in the period immediately around the American Revolution ({\em state government constitution law united power}).
There is also evidence of economic factors: a topic relating to descriptions of cloth ({\em silk cotton ditto white black linen}) rises in the 1750s, but then declines as Americans turned to domestic ``homespun'' cloth production in response to British trade policies.
Other topics point to more subtle changes in language.
A topic that is less immediately interpretable ({\em  say thing might think own did}) corresponds to a series of long ``public letters'' that contain more academic ``argument making''.



Nelson\footnote{Mining the Dispatch, http://dsl.richmond.edu/dispatch/} studies topics in Civil War-era newspapers, including the Confederate paper of record, the Richmond Daily Dispatch.
Like Block and Newman, Nelson's goal is to organize the collection into themes and to measure the variation in prevalence of those themes over time.
The web interface highlights a temporal view of the collection as a series of topic-specific time series.
The mode of analysis is neither fully automated nor manual, but rather combines the two approaches.
Nelson manually labels the topics and groups them into larger categories such as ``slavery'', ``nationalism and patriotism'', ``soldiers'', and ``economy''.

He validates the model by comparing topics to a known and previously annotated category, the ``fugitive slave ads''.
These documents were pre-photographic descriptions of runaway slaves, and have a specific language consisting of aspects of personal appearance and possible locations where enslaved people might have  hidden.
He finds a near perfect correspondence between the prevalence over time of manually labeled fugitive slave ads and documents that have a high concentration of a specific topic, which places high probability on terms such as {\em reward, years,} and {\em color} (manual labels were not used  in training the model).
Nelson notes that few if any of these documents are assigned completely to this topic: he uses a cutoff of 21.5\% as a criterion.

Nelson's larger-scale groupings of topics pick out threads of discourse that may or may not be correlated over time.
The model identifies three topics that have similar temporal distribution, peaking at the beginning of the war in 1861 and largely disappearing afterwards.
These are related but distinct themes: anti-Northern sentiment expressed in poetic form, anti-Northern sentiment expressed in vitriolic prose, and discussion of secession.
All three form aspects of the same process, the rhetorical push for war.
Other related topics have slightly different temporal distributions.
Nelson groups six topics related to soldiers, and displays them in the order of their maximum concentration over time.
They move from ``military recruitment'' and ``orders to report'' to later topics related to ``deserters'', ``casualties'', and ``war prisoners.''
Again, these are related themes but rather than comprising a single event they trace the development of the increasingly dire military situation of the Confederacy.

\citet{yang-11-historical} model a collection of historical newspapers from Texas spanning from the end of the Civil War to the present day.
The goal is both exploratory, to find out about the interests of Texans through the 19th and 20th centuries, and {\em semi-exploratory}, to find out more about the history and context of specific, pre-specified themes such as cotton production.
In the topic model setting, semi-exploratory analysis starts by identifying one or more topics that seem to correspond to the theme of interest, and then using those topics as a axis of investigation into the corpus.
For example, a historian considered documents that exhibit topics related to {\em cotton}, and the topics that co-occur in those documents.
The study also led to more fully exploratory results.
A topic related to the battle of San Jacinto, the final conflict in the Texas Revolution that led to separation from Mexico, appeared earlier than expected.
Further investigation suggested that the significance of the pivotal battle of San Jacinto was established much earlier than historians had previously anticipated.

The Texas newspaper study raises several interesting methodological issues relating to pre-processing and iterative modeling.
The authors put considerable work into dealing with the quality of digitization.
There are many factors that affect the quality of digitized historical newspapers, from the quality of the original printing to scanning, article segmentation, and optical character recognition (OCR).
For this study extensive work was applied to automated spelling correction.
Another notable factor in this study is its prominent use of multiple topic models.
In many cases there is a tacit assumption that a single corpus should result in a single model, but in practice modeling is often iterative, and intimately bound to the development of pre-processing systems.
\cite{yang-11-historical} train different models on different temporal slices of the corpus.
Although there is some advantage to maintaining a consistent topic space over time, dividing the corpus into separate sections has certain advantages.
In this case, historians were interested in the context of specific historic periods, such as the full run of a a newspaper in one of several pivotal years, that are smaller than the full corpus but yet too large to be read easily.
The authors also describe an iterative workflow that involves comparing topic model output after each of several pre-processing steps.
Topic models are often effective at identifying consistent data-preparation errors, such as end-of-line hyphenation and consistent OCR errors.

\section{Historical Records}

Other types of records besides newspapers are of interest, and present their own challenges.
In this section we consider two case studies, in which the simplicity of the bag-of-words document model is an asset because it allows for substantial variability in spelling and language, both in English and in other languages.

\citet{erlin2017topic} search for work related to epistemology in a large corpus of English and German books.
They ``seed'' the models for each language with a small set of query words that the authors expect to be related to that subject.
This approach is closer to standard information retrieval than many other topic model applications, since the model is used both as a way of organizing the corpus and as a way of focusing attention on specific aspects.
Their use of a topic model differs from standard IR in that they are more deliberately open to related terms and concepts: the field of epistemology is expected to be broad, and more likely to be represented by a combination of words than by any one query.

\citet{miller-13} uses Chinese records to investigate the meaning of
the word {\em zei}, or ``bandit'' in Qing dynasty China (1644--1912). The word by
itself can imply several different forms of anti-social behavior,
which are difficult to distinguish from word frequencies alone. A
topic model uses contextual information to separate these effects.

The application of topic models in Chinese highlights the importance of tokenization.
We usually receive documents in the form of long strings, but we are interested in identifying {\em tokens} that are short strings with a specific meaning.
Breaking a document into distinct tokens is an often-overlooked part of the document analysis process.
In European languages we can achieve good results simply by separating strings of letter characters from sequences of non-letter characters, although there are many special cases.
Tokens may contain non-letter characters such as apostrophes and hyphens, and may span multiple words ({\em Queen Victoria}, {\em black hole}).
In many East Asian writing systems we cannot rely on orthographic conventions to identify tokens.
Miller argues that in Classical Chinese a single character can be treated as a token without a strong negative impact on modeling, but for Japanese and modern Chinese we must often rely on pre-processing tools that are themselves potentially unreliable.

%\jbgcomment{Could we get a figure for the months?  I think that would be a nice addition}

Cameron Blevins\footnote{http://www.cameronblevins.org/posts/topic-modeling-martha-ballards-diary/} models the diary of Martha Ballard (1735--1812), a revolutionary war-era midwife who recorded entries over 27 years. The model provides a useful way to discover connections between words and repeated discourses.
As with other historical corpora, Blevins focuses on the connection between topics and time.
Specific events, like a birth, can be highlighted by looking at spikes in a certain topic in the day-to-day time series.
But larger trends are also evident.
As a calibration experiment, Blevins measures the association of a topic that appears to refer to cold weather ({\em cold, windy, chilly, snowy, air}) to months of the year.
As expected, the concentration of this topic is lowest from May to August, rises from September to January, and falls from February to April.

Blevins identifies several other topics that appear to change in their concentration over time.
Two topics involving house work, focusing roughly on cleaning and cooking, respectively, appear to be correlated in time, and rise over the decades.
Blevins connects this finding to suggestions that as Ballard grew older and her children moved away, she had less help from family members.
A more subtle topic involves descriptions of fatigue and illness.
This topic also increases over time, and appears to correlate with the housework topics, except in the last year of the diary, where fatigue and illness reach their highest concentration and housework declines.

This analysis exemplifies the exploratory nature of topic modeling: by themselves, these observations are not conclusive, but they are suggestive and point to areas of further analysis.
A scholar might take the diary entries that score high on an individual topic as a reading list, and determine how well a particular automatically detected discourse maps to themes in Ballard's personal experience.
For example, one might check whether Ballard's references to fatigue and illness are referring to herself or to patients.
The model does not tell the whole story, but it points to where stories might lie.

Blevins argues that characteristics of the diary form make it well-suited for topic analysis: ``Short, content-driven entries that usually touch upon a limited number of topics appear to produce remarkably cohesive and accurate topics.''
In addition, the topic model's lack of linguistic sophistication is actually an asset in this case.
The diary is written in a terse style with many abbreviations and with irregular, 18th century spelling: ``mrss Pages illness Came on at Evng and Shee was Deliverd at 11h of a Son which waid 12 lb.''
Models trained on modern text corpora might not even recognize this example as English, but the topic modeling algorithm is still capable of finding semantically meaningful groups of words.



\section{Scholarly Literature}


%\jbgcomment{The definition of secondary literature seems unclear to me, and it's not clear that the problem of copyright has been discussed enough that it's clear that JSTOR is so great as a result (even though it is!)}

The historical record of scholarship is a valuable source for intellectual history.
Many users make use of the JStor ``Data for Research'' API.\footnote{http://dfr.jstor.org/}
The DFR API is an important example, because it provides access to articles that have been scanned by JStor and may be under copyright.
Access to the underlying documents in their original form as readable sequences of words may be restricted for legal or commercial reasons.
DFR provides a simple view into selected articles by only providing the frequency of word unigrams.
While the bag-of-words assumption used by topic models is restrictive, in this case it can be an advantage, because the original sequence of words is not used for inference anyway.

\cite{mimno-12b} studies a collection of Classics journals digitized by JStor to detect changes in the field over the 20th century.
A distinctive aspect of this study is the use of a {\em polylingual} topic model \cite{mimno-09}.
An English-language journal is compared to a German-language journal by learning a common set of topics that each have a vocabulary in both languages.
In other words, a topic has two ``modes'', one in which it emits words drawn from a distribution over English terms, and another in which it emits words drawn from a distribution over German terms.
The linkage between English and German words is constructed using Wikipedia articles.
Wikipedia articles exist in many different languages, and articles in one language often link to comparable articles in another language.
The author first selects English Wikipedia articles matching key terms in the English-language journals, and then collects the German Wikipedia articles that are listed as being comparable to the selected English-language articles.

By training the topic model jointly on the combined corpus of the original journal articles and the comparable Wikipedia articles, the model provides insight into the relative concentration of scholarly interests across the two language communities.
The German-language journal articles contain relatively more work on law and oratory, themes that are present in the English-language articles but less prevalent.
The model also shows a large increase in  interest in poetry in the German journal in the period following the second world war.
In the English journals there is a large increase starting in the 1980s in cultural and economic studies along with critical theory, which does not appear in the German journals.

\citet{riddell-12} also approaches German scholarly literature from the 20th century. He finds that topics align well with authors such as Goethe and subjects such as folklore. Apparent spikes in the use of these topics appear to align with anniversaries of authors (Goethe, the Grimm brothers).
Riddell emphasizes that models are useful in raising issues but not a substitute for scholarship.
He comments that ``it becomes essential that those using topic models validate the description provided by a topic model by reference to something other than the topic model itself.''

\citet{Goldstone-14} use a topic model as a tool to structure an exploration of a corpus that spans more than a century.
They are interested both in changes at the topic level and at the level of word use within topics.
For these authors the appeal of topic modeling is that models are better able to represent contextual meaning than simple lists of keywords. They write that ``[t]he meanings
of words are shifting and context-dependent. For this reason, it’s risky to
construct groups of words that we imagine are equivalent to some predetermined
concept.''

They analyze the proceedings of the Modern Language Association to find shifts in focus in the field of English literature.
A model trained with 150 topics on 21,000 articles identifies a topic associated with descriptions of violence: {\em power,
violence, fear, blood, death, murder, act, guilt}. Using a temporal plot they argue that the concentration of this topic is greater in the second half of the 20th century than during the first half.
They contextualize this finding by comparing the frequency of these words in a more general corpus from Google ngrams; there is no comparable change.
This approach holds topic fixed and searches for associated words.
They then pivot and hold the word ``power'' fixed and search for associated topics.
In this case the {\em violence} topic actually appears to be relatively stable in its association with the target word. The largest increase is in a different topic characterized by the words {\em own power text form}, in which context it appears almost exclusively after 1980.

In this chapter we have focused on works in which the temporal axis is of primary concern.
When we consider newspapers, historical records, and historical scholarly journals we are looking not just for the topical foci of each time period, but how those topics shift in concentration as they are influenced by historical events.
Our consideration of scholarly journals leads directly into our next chapter on the study of a much larger and temporally variable literature, the study of science.


\chapter{Fiction and Literature}

\label{ch:fiction}

We value literature because it is one of the best ways to capture the spirit of an age, and the experiences of those who lived through it. But standard methods of close reading require focus and thorough interpretation. As a result, scholars are often left trying to make large-scale arguments about the history of literature from small-scale evidence. And this small-scale exploration is not randomly selected: the same small canon is studied in detail while the vast proportion make up the ``great unread'' \citep{moretti-00}, works that are never studied.

\section{The Role Topic Models Play in the Humanities}

As a response, Moretti theorized an alternative ``distant reading''  \citep{moretti-13} that uses computer-assisted methodologies. Topic modeling has emerged as a central tool in distant reading, as a way to organize our reading of large scale patterns.

Topic analysis, viewed as a way of identifying repetitions of language or discourse through multiple works, resonates with many more familiar approaches to the study of literature.
At the broadest scale, to define a genre or a literary period is to separate a corpus into sections based on some observable criterion.
We posit a ``gothic'' literature characterized by atmospheric descriptions of castles, or a ``cyberpunk'' literature characterized by conflicted relationships with information technology.
At a smaller scale, themes or tropes reappear in different contexts.
At the most detailed level, scholars identify repeated phrases, such as the descriptive epithets used in Homeric oral poetry.

Statistical topic analysis has a similar goal, but pursues it through different means.
Rather than rigid boundaries specified by date of publication or nationality, algorithms identify genre through the repeated words that form the traces of those themes.
Topics do not represent themes themselves, but rather identify the implicit statistical regularities in word use brought about by the presence of genres, themes, and discourses.


Applying topic models to fiction, however, brings new challenges. Jockers \citep{jockers-13} trains a 500-topic model on a corpus of 4000 English-language novels.

Several issues emerge from this corpus. These are not unheard of in other contexts, but they are much more readily apparent in fiction.

\section{What is a Document?}

 Treating novels as a single bag of words does not work. We need to find a good segmentation into shorter contexts.

Paragraph-based segmentation [currently trying to track down a reference to work done recently by Stanford Literary Lab]

% Comparison with Twitter?

\section{People and places}

Because works of fiction are set in imaginary worlds that have no existence outside the work itself, they are often characterized by words such as character names that are extremely frequent locally but never occur elsewhere. Modeling these documents can result in topics that are essentially lists of character names.

Jockers and Mimno \citep{jockers-13b} analyze the earlier 500-topic model to determine whether there is a statistically significant connection between the use of specific topics and metadata variables such as author gender, author nationality, and year of publication.

Tangherlini and Leonard \citep{tangherlini-13} look at nested models of sub-corpora within Danish literature.

\section{Beyond the Literal}

One of the hallmarks of fiction and literature is the use of figurative language.
It is not obvious that unintelligent machines with no cultural understanding would have any ability to process such metaphors. However, Rhody \citep{rhody-12} demonstrates on a corpus of poetry that although topics do not represent symbolic meanings, they are a good way of detecting the concrete language associated with repeated metaphors.

Specifically, Rhody explores a corpus of 4500 poems that describe works of art (or {\em ekphrastic} poems).
She trains a 60-topic model, and highlights several particularly interesting topics.
One of these topics places high probability on {\em night, light, moon, stars, day, dark, sun, sleep, sky, wind, time, eyes, star, darkness, bright}.
The apparent meaning of the topic is clear, and well summarized by the single top word: night.
But Rhody finds that when she explores the {\em context} of this topic, the poems are all using a consistent metaphor relating night and sleep to death.
The concept of death does not appear in the top words --- poets are not addressing the issue directly.
Nevertheless, the model has identified an example of non-literal, figurative language even though, because it is grounded in the actual words, it has no ability to represent what the poets actually mean.
The model is able to do this because the poets are using a consistent ``surface'' language to represent a consistent metaphor.
The metaphor is not detectable directly, but a poet's use of a metaphor has a signature that is observable.

Rhody highlights a second topic that provides an example of a different type of non-literal meaning.
This topic places high probability on {\em death, life, heart, dead, long, world, blood, earth, man, soul, men, face, day, pain, die}.
Unlike the previous topic, the topic directly references death and life, but it also lacks what Rhody calls the ``unambiguous comprehensibility'' of the {\em night} topic.
But examining the context of poems that contain the topic reveals a different pattern.
These poems have a consistent {\em form} that Rhody describes as elegiac.
She writes that ``Paul Laurence Dunbar's 'We Wear the Mask' never once mentions the word 'death,' the discourse Dunbar draws from to describe the erasure of identity and the shackles of racial injustice are identified by the model as drawing heavily from language associated with death, loss, and internal turmoil --- language which 'The Starry Night' indisputably also draws from.''




\chapter{Computational Social Science (jordan)}
\label{ch:sci}

\section{Understanding Fields of Studies}

\section{How Fields Change over Time}

\section{Innovation}
% Need to review Justin Grimmer

\chapter{Computational Social Science}
\label{ch:css}

While the previous chapters were mostly retrospective analyses, computational
social science is mostly in the ``here and now''.  It is focus on data
being generated in the past hours, days, or weeks to inform
inteligence analysists, brand monitors, journalists, or social
scientists.  The underlying problem is the same, however: these
stakeholders are interested in what people have to say but cannot read
all of the data at their disposal.

Historically, social science questions such as what candidates is
prefered in a particular part of the country or whether people like a
new restaurant or product were answered by polling: social scientists
would head out into the world, gather a statistically significant
sample, and extrapolate to the broader population.

\jbgcomment{Need to give concrete example of bad press}

These techniques are still valuable, but they still take time.  A
company needs to know if it has an issue with a product immediately,
particularly if its good name is being dragged through the mud on
social media.  However, the reason for the acute time pressure can
also be the solution: if a company is able to quickly see that it has
a social media problem, it can more quickly intervene and correct the
issue.

\section{Sentiment Analysis}

In industrial settings, this problem is often called sentiment
analysis~\citep{pang-08}.  Here, the goal is to determine the
``sentiment''---e.g., positive or negative opinions---associated with
a piece of text.  For example, ``Chipotle is great!'' would be
associated with positive sentiment, while ``Chipotle made me sick
would be associated with negative sentiment''.

While indistrial applications of sentiment analysis is mostly for
identifying whether people like a product or copany, there are wider
social science applications of examining large corpora to determine
authors' \emph{internal state}.  For example, determining whether they
they are politically liberal or conservative based on their online
commentary.

Topic models can help these tasks by dividing a problem into topics.
For example, ``Apple'' can appear in tech news as well as a food
ingredient; someone monitoring the seller of iPods and iPhones would
not want to be confused by social media commentary complaining about a
bad apple pie.  ``surprising'' in an automotive review is likely
associated with negative seintment, while it's a good thing in a book
review.  Thus, topics can help differentiate different kinds of
discussion in broad corpora.

However, topic models lose their value if you want to \emph{contrast}
sentiment within a topic.  While a topic model can find people
discussing Chipotle burritos online, it cannot separate the lovers from
haters.  Thus, \emph{distinguishing} topics based on their sentiment
can help a user better understand how topics and sentiment interact in
a dataset.  This requires modifying the topic model to make it aware
of the underlying sentiment.

% Sentiment is an example of meta data, which can be visualized to
% better understand a corpus (see viz).

\section{Upstream and Downstream Models}

To distinguish topics based on their sentiment, the model must be
aware of what sentiment is.  In the language of probabilistic models,
sentiment and topic are modeled \emph{jointly}.  That is, there is a
probability distribution over both the sentiment of a document $y$ and
the topics that is uses $z$.

There are two general kinds of joint models that incorporate metadata
such as sentiment: upstream and downstream models.  The distinction is
based on the generative story of topic models (Chapter~\ref{}): is
sentiment before (upstream) or after (downstream) topics in the
generative story?

% Put in graphical model examples of upstream and downstream models

Upstream models assume that sentiment comes first.  That is, there
will be different topics given the underlying sentiment.  This can
come in the form in a prior learned from observed
sentiment~\citep{mimno-08} or from a latent variable that can serve as
a proxy for sentiment~\citep{lin-09}.  Upstream models are often
easier to implement and are more flexible~\citep{stewart-14}.

In contrast, downstream models explicitly predict sentiment
\emph{given} text.  If the goal is the later predict sentiment given
raw text with the help of topic models, downstream models can work
better than upstream models.  These models are often called
``supervised'' topic models after supervised
\abr{lda}~\citep{blei-07b}, which 

 and
predictions can be better still with hinge loss~\citep{zhu-09}

These models form the foundation for the models and problems we
discuss in the rest of this section.

\section{Understanding Stance and Polarization}

Often, discussions on an issue can break down into two sides.

Upstream models can discover these sides~\citep{paul-10}

So can downstream models~\citep{nguyen-13:shlda}.

However, there are not always two sides to an issue.

A probabilistic solution to this model is the nested Dirichlet
process~\citep{blei-07}.

These hierarchies induce a non parametric hierarchy over an unbounded
number of topics

This corresponds to agenda setting from political science~\citep{Nguyen:Boyd-Graber:Resnik:Miler-2015}

\section{Topic Models for Understanding Populations}

Traditional social science methods are labor intensive, take a long
time, or are impossible for sensitive subjects

For instance, surveys of influenza take too long to be useful compared
to the life cycle of influenza's progression~\cite{broniatowsky-15}

Monitoring pollution in China or drug use in teens requires access to
populations that may be difficult.  Using social media presents an
alternative~\cite{wang:paul:dredze-15}

% Eisenstein

\section{Social Networks}

We have talked about meta data that are independent for each user.
Sometimes, however, we are interested in meta data that describe the
relationships between people

This makes modeling more difficult, but we still see the same division
between upstream and downstream models

The stochastic block model is the prototype for upstream models~\cite{airoldi-08}

Link LDA is the exemplar for downstream models~\cite{nallapati-08}

Hybrid models can have the best of both worlds

Supervised LDA bases regressions on the topic assignments rather than
the allocations.  Doing something similar can also improve link prediction~\citep{chang-09a}

But changing the objective function can improve performance further~\cite{bach-15}

This can discover geographic variation in language~\cite{eisenstein-10}

But what if we are interested in regions with multiple languages or
dialects?

\chapter{Multilingual Data and Machine Translation}
\label{ch:mt}

So far, we have been focusing on monolingual topic models and their applications.
But many collections contain documents in more than one language.
In practice, we often discover this phenomenon unexpectedly after running an initial monolingual topic model: topic models turn out to be very good at language identification.
This behavior makes sense because the model is looking for groups of words that appear frequently together but not in other contexts, and separate languages will exhibit this property strongly.
In some cases we may choose to filter out small numbers of documents in other languages, but in many cases we would like to take advantage of  connections across many languages.

Multilingual topic models have been developed to analyze and understand a corpus in multiple languages.
The applications in multi-language corpora can be divided into two categories.
The first, and simpler category is those that aligning languages at the topical level, but not at the level of individual word types.
These models are useful for organizing corpora, but make no attempt to support analysis for users unfamiliar with any particular language.
The second category is those that explicitly model word-level alignments across languages.
These models support applications in statistical machine translation~(\textsc{smt}).

This chapter is organized into two sections. The first describes models that can be applied to multiple languages. The second focuses on a specific application, statistical machine translation.

Before discussing specific methods, it is useful to define terms related to data sources.
The most salient feature for multilingual corpora is their degree of alignment.
Parallel corpora are the most closely aligned. These collections comprise subsets of documents such that each set contains documents in different languages that have the same semantic content (up to the limits of translation).
Common examples include translations of literary works or translated government documents, where a transcript of a speech in French is accompanied by a transcript of the  same speech in German, with as little semantic difference as possible.
Comparable corpora are less closely aligned. 
These collections also contain subsets of documents, but each set is only constrained to be {\em topically} similar, and not necessarily a direct translation.
A common example is articles in Wikipedia.
The articles for the French city of Lille in English and French Wikipedia are referring to the same place and contain much of the same information, but the French version is considerably longer.
Mixed corpora are the least aligned. These collections simply contain documents in more than one language, but there is not necessarily any connection between any one document in one language and a document in another language.
An example might be a journal that publishes in English, French, German, and Italian.
No article is a translation of any other article.
There are likely to be topical overlaps between articles, but there are not necessarily any structural indications of such relationships.
A last category of useful data, not necessarily in the form of documents, is a bilingual lexicon that maps words in one language to words in another.
Lexicons of this form can be considered to be examples of parallel corpora with single-token documents, but it is often useful to treat them specially.

\section{Document-level Alignment from Multilingual Corpora}

The weaker category of multilingual models represents the presence of multiple languages, but does not explicitly represent connections between word types.
Document-based connect connection between languages is flexible: instead of requiring
the exact matching or translations on words and sentences, only a
coarse document alignment is necessary, as long as the documents
discuss the same topics, e.g., wikipedia articles in different
languages. Such connection between languages is also helpful to infer
more robust topics, since different languages can complement each
other to reduce ambiguity.

This approach pre-dates probabilistic topic models.
\citet{Landauer-1990} connect aligned documents in different languages
by projecting both documents to a shared latent semantic indexing
space.

Similarly, bilingual \citep{zhao-06} and polylingual topic models~\citep[\plda{}]{mimno-09}
assume that the aligned documents in different languages share the
same topic distribution and each language has a unique topic
distribution over its word types.
Thus the generative process of polylingual topic model is as follows:
given a document pair $(d_{l_1}, d_{l_2})$, we first sample a
document-topic distribution $\theta_d$; for a document $d_{l_i}$ in
language $l_i$, we then sample a topic $z_{dn}$ from $\theta_d$, and
generate a word from topic $\phi_{z_{dn}, l_i}$ in language $l_i$.

Topic models trained from document-level alignments have applications in exploratory data analysis and in information retrieval.
\cite{mimno-09} uses a model trained on multiple languages in Wikipedia to compare relative interest in different topics across linguistic domains.
For example, the Farsi-language Wikipedia has a larger than average number of articles about science, while the Finnish-language Wikipedia has a larger than average number of articles about skiing.
These methods require parallel or comparable corpora, but for mixed corpora the training data can be augmented with a supplemental corpus of comparable documents as long as  the comparable documents cover similar enough topics \citep{mimno-12b}.
 
We note that it is not necessary to take ``language'' in its strict meaning.
Loosely aligned models have been applied in information retrieval for query expansion \citep{Gao-2011,Gao-2012}. They assume the search
query and its relevant web documents share a common distribution of
topics, but use different vocabularies to express these topics. Thus
in their models, queries and documents share the same document-topic
distributions $\theta^Q$, but have different topic-word distributions
$\phi_z^Q$ and $\phi_z^D$ respectively.
%To generate a query $q$, a document-topic distribution $\theta^Q$ is
%drawn from a Dirichlet prior, and a topic $z$ is sampled from
%$\theta^Q$, then a query term $q_i$ is sampled from $\phi_z^Q$. To
%generate a document term, a topic $z$ is firstly sampled from
%$\theta^Q$, the same document-topic distribution as the query, and
%then a document term $d_i$ is sampled from document topic-word distribution $\phi_z^D$. 
In this way, documents and queries are
connected through the hidden topics, even though their vocabularies
(topic-word distributions) are different. By summing over all possible
topics, the relationship between document term $e$ and query $q$ can
be computed as,
\begin{align}
p(e|q) = \sum_k p(e|\phi_k^D) p(k | \theta_q).
\end{align}

Some forms of query expansion across multiple languages do not require explicit modeling of connections between topics.
\cite{erlin2017topic} uses two independent seeded models on English and German books to search for works about epistemology.
After manually aligning one topic from each model, the topics are used as a ``soft query'' to identify words that are related to documents but the target subject.

\section{Word-level Alignment from  Lexical Data}

Non-document \emph{lexical information}, such as orthographic
similarity~\citep{boyd-graber-09} and multilingual
dictionaries~\citep{boyd-graber-10}, can be helpful in learning
better topics from multilingual corpora. For instance, tree-based
topic models~\citep[\tlda{}]{boyd-graber-07,andrzejewski-09,hu-14:itm}
incorporate positive correlations between words in the same or
different languages by encouraging words that appear together in a
{\bf concept} to have similar probabilities given a topic. These
concepts can come from WordNet~\citep{boyd-graber-10}, domain
experts~\citep{andrzejewski-09}, or user
constrains~\citep{hu-14:itm}. If these concepts are in the same
language, the backend model is the same as monolingual interactive topic modeling
introduced in Chapter~\ref{ch:viz}. However, when we gather concepts
from bilingual resources, these concepts can connect different
languages. For example, if a bilingual dictionary defines
``\begin{CJK*}{UTF8}{gbsn}电脑\end{CJK*}'' as ``computer'', we combine
  these words in a concept.

These concepts (positive correlations) are organized into a {\bf prior
  tree} structure. As Figure~\ref{fig:prior_trees} shows, words in the
same concept share a common parent node. That concept then becomes
one of many children of the root node.  Words that are not in any
concept---{\bf uncorrelated words}---are directly connected to the
root node. Thus a topic becomes a distribution over all paths in this
prior tree and each path is associated with a word.

\begin{figure}
\centering
\includegraphics[width=0.9\linewidth]{figures/correlations_tree-crop.pdf}
\vspace{-3mm}
\caption[Constructing prior tree from a bilingual dictionary]{An example of constructing a prior tree from a
  bilingual dictionary: word pairs with the same meaning but in
  different languages are concepts; a common parent node is created to
  group words in a concept, and then is connected to the root;
  uncorrelated words are connected to the root directly.}
\label{fig:prior_trees}
\end{figure}

The probability of a path in a topic depends on the transition
probabilities in a topic.  Each concept $i$ in topic $k$ has a
distribution over its child nodes that is governed by a Dirichlet prior:
$\pi_{k,i} \sim \text{Dir}(\beta_{i})$.  Each path ends in a word
(i.e., a leaf node) and the probability of a path is the product of
all of the transitions between topics it traverses. Topics have
correlations over words because the Dirichlet parameters can encode
positive or negative correlations~\citep{andrzejewski-09}.

As a result, to sample a word $w_{dn}$ given a topic $z_{dn}$, a path
$y_{dn}$ from the topic tree of topic $z_{dn}$ is sampled: we start
from the root $n_0$ and first sample a child node $n_1$ of the root;
if node $n_1$ is a concept node, we continue to sample a word node
$n_2$ and generate the word associated with $n_2$; if node $n_1$ is a
word node already, we generate the word directly.

\jbgcomment{Give intuition, e.g. selecting a concept first and then
  you get the language-specific version. \yhcomment{added.}}

When this tree serves as a prior for topic models, words in the same
concept are positively correlated in topics.  For example, if
``\begin{CJK*}{UTF8}{gbsn}电脑\end{CJK*}'' has high probability in a
  topic, so will ``computer'', since they share the same parent
  node. With the tree priors, each topic is no longer a distribution
  over word types; instead, it is a distribution over paths, and each
  path is associated with a word type.  The same word could appear in
  multiple paths, and each path represents a unique sense of this
  word.



\jbgcomment{Would be nice if the prior tree were multilingual.  Would
  also be good to give back pointer to the interaction section where
  it's also discussed \yhcomment{fixed.}}

\section{Alignment from Parallel Corpora and Lexical Information}

Bilingual dictionaries and other sources of word-level information are 
valuable in training multilingual models, because they can easily specify 
simple lexical relationships that might be difficult to extract from parallel corpora.
But such manually generated data may be brittle, low-quality, or missing contextual differences in actual usage.
These two approaches are not mutually exclusive, however; they reveal
different connections across languages. \citet{hu-14} bring existing
tree-based topic models~(\tlda{}) and polylingual topic
models~(\plda{}) together and create the polylingual tree-based topic
model~(\ptlda{}) that incorporates both word-level correlations and
document-level alignment information.

To build up the prior tree structure, \citet{hu-14} consider two
resources that correlate words across languages. The first is
multilingual dictionaries, which match words with the same meaning in
different languages together. The other is the word alignments
extracted from aligned sentences in a parallel corpus. These relations
between words are used as the concepts~\citep{Bhattacharya-2006} in
the prior tree (Figure~\ref{fig:prior_trees}).

Given the prior tree structure, the generation of documents is a
combination of \tlda{} and \plda{}.  For each aligned document pair
$(d_{l_1}, d_{l_2})$, we first sample a distribution over topics
$\theta_d$ from a Dirichlet prior $\text{Dir}(\alpha)$.  For each
token in the aligned document $d_{l_i}$, we first sample a topic
$z_{dn}$ from the multinomial distribution $\theta_d$, and then sample
a path $y_{dn}$ along the tree of topic $z_{dn}$. Because every path
$y_{dn}$ leads to a word $w_{dn}$ in language $l_{dn}$, we append the
sampled word $w_{dn}$ to document $d_{l_{dn}}$ in language $l_{dn}$.

If a flat symmetric Dirichlet prior is used instead of the tree prior,
\plda{} is recovered; and if all documents are monolingual (i.e., with
distinct distributions over topics $\theta$), \tlda{} is
recovered. \ptlda{} connects different languages on both the word
level (using the word correlations) and the document level (using the
document alignments), thus it learns better topics by considering more
information from both languages.


\section{Topic Models and Machine Translation}

The most frequent application of multilingual topic models is in machine translation.
Given a text input in one language (source language), statistical
machine translation tries to find a similar sequence of words in another
language (target language). Modern machine translation
systems~\citep{koehn-09} use millions of training examples to learn
the translation rules and apply these rules on the test data. 
Topic models are useful in this application when they can help to inform word meaning and word choice in specific contexts.
While the translation rules are learned in local context, these systems work
best when the training corpus has a consistent \emph{domain}, such as a
 genre (e.g., sports, business) or style (e.g.,
newswire, blog-posts). 

Translations within one domain are better than translations across
domains since they vary dramatically in their word choices and style.
A correct translation in one domain may be inappropriate in another
domain.  For example, ``\begin{CJK*}{UTF8}{gbsn}潜水\end{CJK*}'' in the
  \underline{sports} domain usually means ``underwater diving'', but
  in the \underline{social media} domain, it means a non-contributing
  ``lurker''. To avoid such translation errors caused by a domain
  change, domain knowledge is needed to train translation systems that
  are robust to such systematic variation in the training set, which
  are said to exhibit \emph{domain adaptation}.

To train such \textsc{smt} systems with domain adaptation, early
efforts focus on building separate models given the hand-labeled
domains~\citep{foster-07,matsoukas-09,chiang-11}. However, this setup
is at best expensive and at worst infeasible for large data.  Topic
models provide a promising solution where domains can be automatically
discovered. Each extracted topic is treated as a soft
domain.\footnote{Henceforth we will use the term ``topic'' and
  ``domain'' interchangeably: ``topic'' to refer to  a word distribution in
  topic models and ``domain'' to refer to \textsc{smt} corpora.} Thus
the normal monolingual topic models trained only on the source documents have
been applied to extract domain knowledge for machine
translation~\citep{Eidelman-12}.

However, the source language the and target language can complement
each other to build up more accurate topic models. For example, if we
only know the Chinese phrase ``\begin{CJK*}{UTF8}{gbsn}潜
  水\end{CJK*}'', it is hard to decide whether it is a
  \underline{sport} domain or it is a \underline{social media}
  domain. However, with the help of the aligned English translation
  ``lurker'', it is easy to identify the ``social media'' domain. Thus
  multilingual topic models~\citep{ni-09,DeSmet-09,mimno-09,boyd-graber-10} have been
  applied to extract domain knowledge for machine
  translation~\citep{hu-14}.

\section{The Components of Statistical Machine Translation}

Statistical machine translation represents translation as a
combination of probabilistic processes, a phrase-level translation model and a sentence-level language model~\citep{koehn-03,koehn-09}. 
Topic models have been applied to both aspects of this process.
Given a source sentence $\mathbf{f}$, the best
translation in target language $\mathbf{e}_\texttt{best}$ is
\begin{equation}
\mathbf{e}_\texttt{best} = \textbf{argmax}_\mathbf{e} p(\mathbf{e}|\mathbf{f}) = \textbf{argmax}_\mathbf{e} p(\mathbf{f}|\mathbf{e}) p (\mathbf{e}),
\end{equation}
which is split to a \textit{translation model}
$p(\mathbf{f}|\mathbf{e})$ and a \textit{language model} $p
(\mathbf{e})$.
Intuitively, a good translation should be both a good match for the source sentence (scoring high in the translation model) and a good sentence in its own right (scoring high in the language model).

The source sentence $\mathbf{f}$ is segmented into multiple source
phrases $\bar{f}_n$ during the decoding, which are translated to a set
of target phrases $\bar{e}_n$. Thus the translation probability
$p(\mathbf{f}|\mathbf{e})$ can be further decomposed to the phrase
translation probability $p(\bar{f}_n | \bar{e}_n)$. Target phrases may then need to be \textit{reordered} to get the best
translation result. Reordering is captured by a relative distortion
probability distribution $d(a_n - b_{n-1})$, where $a_i$ denotes the
start position of the source phrase that was translated to the $n$th
target phrase, and $b_{n-1}$ denotes the end position of the source
phrase translated into the $(n-1)$th target phrase. As a result, the
translation model can be decomposed as,
\begin{equation}
p(\mathbf{f}|\mathbf{e}) = \prod_{n} p(\bar{f}_n | \bar{e}_n) d(a_n - b_{n-1})
\end{equation}

In phrase-based \textsc{smt}, the phrase probability $p(\bar{f}_n |
\bar{e}_n)$ can be further estimated by combining lexical translation
probabilities of words contained in that phrase~\citep{koehn-03},
which is normally referred as \textit{lexical weighting}. Lexical
conditional probabilities $p_w(f|e)$ are maximum likelihood estimates
from relative lexical frequencies,
\begin{equation}
\label{eq:lexical_prob}
p_w(f|e) = \textstyle \slfrac{c(f, e)}{\sum_f{c(f, e)}}
\end{equation}
where $c(f, e)$ is the count of observing lexical pair $(f, e)$ in the
training dataset. Given a word alignment $a$, the lexical weight for
this phrase pair $p_w(\bar{f} | \bar{e}; a)$ is the normalized product
of lexical probabilities of the aligned word pairs within that phrase
pair:
\begin{equation}
\label{eq:phrase_prob}
p_w(\bar{f} | \bar{e}; a) = \prod_{i} \frac{1}{\{|j | (i, j) \in a\}|} \sum_{\forall (i,j) \in a} p_w(f_i | e_j)
\end{equation}
where $i$ and $j$ are the word positions in target phrase $\bar{e}$
and source phrase $\bar{f}$ respectively.

Next we introduce how to apply topic models to improve translation
models, language models, and reordering models respectively.


\section{Topic Models for Phrase-level Translation}
\label{sec:trans-multiling}

Translation models map words and phrases from one language to another.
Both monolingual topic models and bilingual topic models are useful for improving translation models.
The most prominent application of topic models is in {\em domain adaptation}.

To train such \textsc{smt} systems with domain adaptation, early
efforts focus on building separate models based on the hand-labeled
domains~\citep{foster-07,matsoukas-09,chiang-11}. For example, for all
the training examples that are labeled as \underline{sports} domain,
one translation model is trained. As a result, in any test example
that is labeled as \underline{sports}, ``\begin{CJK*}{UTF8}{gbsn}潜
  水\end{CJK*}'' is always translated to ``underwater diving'', and
  the probability of translating ``\begin{CJK*}{UTF8}{gbsn}潜
    水\end{CJK*}'' to ``lurker'' is zero. In fact, such hard domain
    labels are not only expensive and time consuming to obtain, but
    also unsmoothed and sensitive to labeling errors.

Hard domain labels are difficult to apply and can decrease the robustness of translations: domains are fundamentally uncertain, and if you get the domain wrong, you may cut off useful information.
Topic models provide a way of automatically discovering soft domain
assignments. 
If we equate the $K$ topic distributions over the vocabulary in a topic model with $K$ 
\textsc{smt} domains, each document's topic distribution can be viewed as a soft
domain assignment for that document.
If there
are two topics \underline{sports} and \underline{social media} and
a test example is most likely about \underline{sports}, it may
have a soft domain distribution as $85\%$ for \underline{sports}
domain and $15\%$ for \underline{social media} domain. These automatically
obtained soft domain labels are well smoothed, and they are not only
cheap to obtain but also much more robust to topic errors. 
We next describe applications of monolingual and multilingual
topic models to improve translation models~\citep{Eidelman-12,hu-14}.

\jbgcomment{Give an example of Chiang's hard domain assignment and
  then give an example of how topic models can do this using ``soft''
  assignment.  Then go into the mathematical
  details. \yhcomment{added.}}

%\subsection{Translation Domain Adaptation with Topic Models}


%Given the soft domain assignments, \citet{Eidelman-12} extract
%lexical weighting features conditioned on the topics, optimizing
%feature weights using the \emph{Margin Infused Relaxed
%Algorithm}~\cite[\textsc{mira}]{Crammer:2006}.  The topics come from
%source documents \emph{only} and create topic-specific lexical
%weights from the per-document topic distribution $p(k|d)$, which is
%used to smooth the expected count $\hat{c}_{k}(f,e)$ of a word
%translation pair under topic $k$,

\paragraph{Translations from monolingual topic models}

We can train a translation model by counting the frequency of pairs from word-level alignment data.
\citet{Eidelman-12} builds topic-specific translation models by reweighting the frequency of word pairs based on soft topic/domain assignments for documents.
Since a translated document is assumed to have the same topics in both languages, we only require a monolingual topic model trained on one or the other language.
The document-topic distribution $p(k|d)$ is used
to smooth the expected count $\hat{c}_{k}(f,e)$ of a word translation
pair under topic $k$,
\begin{align}
\textstyle \hat{c}_{k}(f,e) = \sum_{d}{p(k|d)c_d(f,e)},
\end{align}
where $c_d(\bullet)$ is the number of occurrences of the word pair in
document $d$.  The lexical probability conditioned on topic $k$ is the
unsmoothed probability estimate of those expected counts
\begin{align}
\label{eq:lexical_prob_k}
\textstyle p_w(f|e;k) = \hat{c}_{k}(f,e) / \sum_f{\hat{c}_{k}(f,e)},
\end{align}
from which we can compute the lexical weight of this phrase pair
$p_w(\bar{f}|\bar{e};a, k)$ given a word alignment $a$ \citep{koehn-03}:
\begin{align}
\label{eq:phrase_prob_k}
p_w(\bar{f} | \bar{e};a, k) = \prod^{n}_{i=1} \frac{1}{\{|j | (i, j) \in a\}|} \sum_{\forall (i,j) \in a} p_w(f_i | e_j; k)
\end{align}
where $i$ and $j$ are the word positions in target phrase $\bar{e}$
and source phrase $\bar{f}$ respectively. 
Equations~\ref{eq:lexical_prob_k} and \ref{eq:phrase_prob_k} are equivalent to Equations~\ref{eq:lexical_prob} and ~\ref{eq:phrase_prob}, but with the addition of soft topic/domain assignments.
\citet{Eidelman-12} combines the standard $f(\bar{f}|\bar{e})$ and
$f(\bar{e}|\bar{f})$ with two directions of
topic-adapted probabilities 
$p_w(\bar{f} | \bar{e};a, k)$ and $p_w(\bar{e} | \bar{f};a, k)$, equivalent to introducing $2K$ new word translation tables. 
Feature weights are optimized through
using the Margin Infused Relaxed
  Algorithm (MIRA)~\cite[\textsc{mira}]{Crammer-06}.

For a test document $d$, the document topic distribution $p(k | d)$ is
inferred based on the topics learned from training data. The lexical
weight feature of a phrase pair $(\bar{f}, \bar{e})$ is,
\begin{align}
\textstyle f_{k}(\bar{f}|\bar{e})=-\log\left\{{p_{w}(\bar{f}|\bar{e};k)\cdot p(k|d)}\right\},
\end{align}
a combination of the topic dependent lexical weight and the topic
distribution of the document, from which we extract the phrase.

These adapted features allow us to bias the translations according to
the topics. For example, if topic $k$ is dominant in a test document,
the feature $f_k(\bar{f} | \bar{e})$ will be large, which may bias the
decoder to a translation that has small value of the standard feature
$f(\bar{f}|\bar{e})$. In addition, combining the adapted features with
the standard features makes this model more flexible. For a test
document with less clear topics, the topic distribution will tend
toward being fairly uniform. In this case, the topic features will
contribute less to the translation results and the standard features
will dominate the translation results.

%\paragraph{Topical Lexical and Phrasal Features}

%\citet{hasler-12} also apply monolingual topic models for domain adaptation to
%\textsc{smt} in a similar framework as \citet{Eidelman-12}, except
%they introduce different features which they call sparse word pair
%features and phrase pair features. The topics on source documents are
%integrated as a source side trigger for a particular word pair or
%phrase pair as sparse features. For example, given a word $w_f$ with
%topic $k$, for an aligned word pair $(w_f, w_e)$ which is observed $c$
%times in the aligned sentence pair, the sparse word pair feature $wp$
%with topics is represented as $wp\_k\_w_f \sim w_e = c$, while the
%original word pair feature is $wp\_w_f \sim w_e = c$. The topic phrase
%pair features are also defined in a similar way: given an aligned
%phrase pair $(p_f, p_e)$ with count $c$ in the same sentence, the
%sparse phrase pair feature $pp$ with topics is represented as
%$pp\_k\_p_f \sim p_e = c$. Both the phrase pair and word pair features
%are extracted from the aligned training sentence pairs, and then
%\textsc{mira} is used to learn the feature weights.
%
%One difference in \citet{hasler-12} from \citet{Eidelman-12} is that
%they are using \emph{hidden topic Markov
%  models}~\citep[\textsc{htmm}]{gruber-07} instead of \textsc{lda} to
%learn topics. While \textsc{lda} assumes that each word is generated
%independently in a document, \textsc{htmm} models the word topic in a
%document as a Markov chain where all words in a sentence are assigned
%with the same topic. \textsc{htmm} computes $p(z_n, \Phi_n | d, w_1,
%\cdots, w_N)$ for each sentence, where $z_n$ is the topic of sentence
%$n$ in document $d$, $w_1, \cdots, w_N$ are the words in the sentence
%$n$, $\Phi_n$ is the topic transition between words. $\Phi_n$ is only
%non-zero at sentence boundaries. The advantage of using \textsc{htmm}
%is that each sentence gets the same topic assignment, thus the topic
%for each phrase pair in the aligned sentence is consistent and can be
%used for topical features directly.


\citet{hasler-12} also apply monolingual topic models for domain adaptation to
\textsc{smt} in a similar framework as \citet{Eidelman-12}, except
they apply \emph{hidden topic Markov models}~\citep[\textsc{htmm}]{gruber-07} 
instead of \textsc{lda} to learn topics and extract different features. 
While \textsc{lda} assumes that each word is generated
independently in a document, \textsc{htmm} models the word topic in a
document as a Markov chain where all words in a sentence are assigned
with the same topic. As a result, the topic
for each phrase pair in the aligned sentence is consistent and can be
used for topical features directly.

%\paragraph{Topical Phrase Probability via Topic Mapping}

\citet{su-12} use \textsc{htmm} to incorporate topic information
into the phrase probability directly, rather than through the word
translation probability.
Given the bilingual translation
training data without any specific domain information (referred as
out-of-domain bilingual data), they incorporate topic information
from the source language into translation probability estimation, and
decompose the phrase probability $p(\bar{e}|\bar{f})$ as,
\begin{align}
p(\bar{e}|\bar{f}) = \sum_{k_{out}} p(\bar{e}, k_{out} | \bar{f}) = \sum_{k_{out}} p(\bar{e} | \bar{f}, k_{out})  \cdot p(k_{out} | \bar{f})
\end{align}
where $p(\bar{e} | \bar{f}, k_{out})$ is the translation probability given 
the source side topic $k_{out}$, and $p(k_{out} | \bar{f})$ denotes the 
phrase probability in topic $k_{out}$.

Besides, \citet{su-12} assume that a monolingual corpus in the
same domain as the test sentence (referred to as in-domain monolingual
data) is available. Thus they also apply \textsc{htmm} to estimate the
in-domain topic $k_{in}$ and $p(k_{in} | \bar{f})$.  However, the
in-domain topics $k_{in}$ and the out-of-domain topics $k_{out}$ may
not be in the same space, so \citet{su-12} introduce the topic mapping
probability $p(k_{out} | k_{in})$ to map the in-domain topic to the
out-of-domain topic:
\begin{align}
p(k_{out} | \bar{f}) = \sum_{k_{in}} p(k_{out} | k_{in}) \cdot  p(k_{in} | \bar{f})
\end{align}
As a result, the final phrase probability can be refined as,
\begin{align}
p(\bar{e}|\bar{f}) = \sum_{k_{out}} \sum_{k_{in}} p(\bar{e} | \bar{f}, k_{out}) \cdot p(k_{out} | k_{in}) \cdot p(k_{in} | \bar{f})
\end{align}
The topic and topic mapping relationship
between the training data and test data can be built offline, so the
whole process adds no additional burden to the translation system.

Monolingual topic models can add contextual information about word choice to 
translation models, but do not by themselves take advantage of multilingual information.
We next turn to topic models that explicitly learn multilingual connections between words.

\paragraph{Multilingual Information for Domain Adaptation}

Using bilingual data adds modeling complexity, but can also improve topic model quality.
One can think of topic models as tools for disambiguating the meaning of words based on their context. 
Aligning across multiple languages is a common way of resolving such ambiguities.
 For example, ``\begin{CJK*}{UTF8}{gbsn}木马\end{CJK*}'' in a Chinese
  document can be either ``hobbyhorse'' in a \underline{children}'s
  topic, or ``Trojan virus'' in a \underline{technology} topic.  
  A monolingual topic model might not be able to tell the difference in a short Chinese document, but these terms
  are unambiguous in English, more accurately indicating the relevant topic.

Multilingual topic models have been shown to improve results in \textsc{smt} \citep{hu-14}.
There are various ways to build up the multilingual topic
models. Different languages can be connected on the
word-level~\citep{boyd-graber-07,andrzejewski-09,hu-14:itm} or the
document levels~\citep{mimno-09}, or a combination of both \citep{hu-14}. 

While the above approaches try to model the source and target
languages simultaneously to extract topics, some of the benefit of multilingual models can be achieved by aligning monolingual models. \citet{xiao-12} apply
topic models on the source documents and target documents separately
to learn the document-topic distributions $p(k_f | d_f)$ and $p(k_e |
d_e)$, and then estimate the phrase-topic probabilities $p(\bar{e}, k_f
| \bar{f})$ and $p(\bar{e}, k_e | \bar{f})$ from each model. They further
compute the topic similarity scores between the phrase topic
distribution and document topic distribution as features for decoding
to improve \textsc{smt} results.

To translate a new document $d_f$, they first estimate the document-topic distribution $p(k_f |
d_f)$.
Then for a given phrase $\bar{f}$ in the source document they search for the target phrase $\bar{e}$ that maximizes the similarity between the source document's topic distribution $p(k_f|d_f)$ and the phrase-topic distribution $p(\bar{e}, k_f|\bar{f})$ according to squared Hellinger distance $H^2(p,q) = \sum_k \left(\sqrt{p_k} - \sqrt{q_k} \right)^2$.
Second, they calculate a projection between the two monolingual topic models $p(k_f
| k_e)$  by normalizing the co-occurrence count in the aligned training
sentences, and use this relationship to calculate the conditional distribution of target phrases and target topics $p(\bar{e}, k_e|\bar{f})$.

This topic projection idea is similar to the topic mapping by
\citet{su-12}, but it is applied between the source language and the
target language. Compared to the lexical features in
\citet{Eidelman-12} and \citet{hu-14}, \citet{xiao-12} introduce a new
framework to apply topic information directly to measure the relationship between phrases and present two topic similarity features for decoding. These two approaches can
be combined to further improve \textsc{smt}.

\section{Topic Models for Sentence-level Language Modeling}


A critical component of machine translation systems is the language
models, which provide local constraints and preferences to make
translations more coherent. A language model describes the probability
of a word $w$ occurring given the previous context words, which is
also mentioned as the history $h$. We have introduced the application
of language models in information retrieval in
Chapter~\ref{sec:ir-lm}. In fact, it also helps to choose the correct
or more proper word during the statistical machine translation. For
example, the English words ``house'' and ``home'' are in many cases synonymous, but the translation ``I am going home'' is better than
``I am going house''.

Domain adaptation for language models~\citep{Bellegarda-04,wood-09}
uses extra knowledge to adjust this probability $p(w|h)$ to reflect
the content change, which is an important avenue for improving machine
translation. As \citet{Bellegarda-04} points out, ``an adaptive
language model seeks to maintain an adequate representation of the
current task domain under changing conditions involving potential
variations in vocabulary, syntax, content, and style''.

Topics from topic models can be one of the resources to provide such
knowledge for language model adaptation. For example, the Chinese
phrase ``\begin{CJK*}{UTF8}{gbsn}很多粉丝\end{CJK*}'' is translated to
  ``a lot of vermicelli'' in a \underline{food} domain, but means ``a
  lot of fans'' in an \underline{entertainment} domain. Such ambiguity
  can be reduced by using topic/domain knowledge. If the \underline{entertainment} topic is extracted based on
  the previous context, this Chinese phrase will be translated to ``a
  lot of fans'' without any ambiguity. Next, we introduce the details
  about applying topic models for language model adaptation.

\paragraph{Language Model Adaptation from Monolingual topic models}

Early work~\citep{Clarkson-1997,Seymore-1997,Kneser-1997,Iyer-1999} focuses on
partitioning the training data to multiple topic-specific subsets and
building up language models for each subset. Then the topic-specific
language models $p_k(w|h)$ are linearly combined with a general
language model $p_g(w|h)$ built from all training data as
Equation~\ref{eq:linear_lm}. The weights $\lambda_k$ can be tuned
based on the topics of the test documents.
\begin{align}
\label{eq:linear_lm}
p_\textrm{adapted}(w|h) = \sum_k \lambda_k p_k(w|h) + \lambda_g p_g(w|h)
\end{align}

\citet{Seymore-1998} further identify the most appropriate topic for
each word in the vocabulary and choose either a topic-specific language model
or the general language model. The intuition is that the general
language model provides the most reliable estimation for general
words, and the topic language model estimates the probability more
accurately for more specific words. As a result, they split the vocabulary
words into three groups: the general subset, on-topic subset and
off-topic subsets. They use the general language model for the general subset and the off-topic subset and the topic-specific language model for  the
on-topic subset.

All of these methods use a traditional n-gram model, which conditions on a finite, bounded history.
These models also assume each document or history belongs
to exactly one topic cluster.
To fix these problems, models with topic mixtures, such as
\emph{Latent Semantic Analysis}~\citep[\textsc{lsa}]{deerwester-90}
and its probabilistic interpretation probabilistic latent semantic
indexing (pLSI)~\citep[\textsc{plsi}]{hofmann-99},
learn large-span language
models~\citep{Bellegarda-1997,Coccaro-1998,Gildea-1999}. \citet{Gildea-1999}
decomposes the language model based on topics,
\begin{align}
\label{eq:plsi_lm}
p(w|h) = \sum_k p(w|k) p(k|h)
\end{align}
where the topics are learned from the training corpus by optimizing the log probability,
\begin{align}
l(\theta; N) = \sum_w \sum_d n(w,d) \log \sum_k p(w|k) p(k|d)
\end{align}
where $d$ is the training documents, and $n(w,d)$ is the word frequency of $w$ in document $d$. $p(w|k)$ and $p(k|d)$ are learned through the EM algorithm. For test documents, they fix $p(w|k)$ to estimate $p(k|h)$ and then compute $p(w|h)$ using Equation~\ref{eq:plsi_lm}. 

This way of applying topic models to language models is 
similar to document language modeling for information retrieval
as in Chapter~\ref{sec:ir-lm}. However, unlike
information retrieval, two different languages are involved in the
process of \textsc{smt}, and they can complement each other to learn
more accurate topics. Next, we discuss multilingual topic models
for language model adaptation.

\paragraph{Language Model Adaptation from Multilingual Topic Models}

As we explained in Section~\ref{sec:trans-multiling}, the information
from different languages can complement each other to extract better
topics. Latent semantic models such as LSA have been used in multilingual information retrieval for many years \cite{carbonell1997translingual}.
In order to introduce multilingual information to probabilistic topic models
for language model adaptation, various approaches such as
\cite{Tam-2007}, \cite{Ruiz-2011}, \cite{Yu-2013} etc., have been
explored. 

\cite{Tam-2007} introduces the bilingual latent semantic
analysis~(\textsc{blsa}) to learn the topics for both source language
and target language, and apply the learned topics into language model
adaptation for \textsc{smt}. Similar to polylingual topic models~\citep{mimno-09}, 
\textsc{blsa} transfers the inferred
topics from the source language to the parallel target language.

More specifically, \cite{Tam-2007} assume the aligned source document
and the target document share the same document-topic distribution.
They first learn an \textsc{lsa} model on the source
language and then use the document-topic vector from the
source document as the document-topic vector for the aligned
target document, and then infer the topic-word vector on the
target side. The topics for the target language are not learned
iteratively, thus the topics in a parallel corpus can be learned very
efficiently.

In order to apply the topics for language adaptation, the word
marginal distribution $p_{lsa}(w)$ for document $d$ is computed,
\begin{align}
p_{lsa}(w) = \sum_{k=1}^K p(w|k) p(k|d)
\end{align}
Then this word marginal distribution is integrated into the target background language model by minimizing the KL divergence between the adapted language model and the background language model~\citep{Kneser-1997b}:
\begin{align}
\label{eq:mdi}
p_a(w|h) \propto \Big( \frac{p_{lsa}(w)}{p_{bg}(w)} \Big) ^{\beta} \cdot p_{bg}(w|h)
\end{align}

\cite{Ruiz-2011} apply a similar idea for language model
adaptation. Instead of using \textsc{blsa}, \cite{Ruiz-2011} simply
merge the aligned source and target document as one document, and
train pLSI. Both ideas are
based on the assumption that the aligned source document and target
document share the same document-topic distribution. The final adapted
language model combines the topic-based language model with the
general background language model, thus it is more robust in improving
the results of \textsc{smt}.

\jbgcomment{again, good to have ex}

\citet{Yu-2013} presents a hidden topic Markov model~(\textsc{htmm}) to improve
the language model in \textsc{smt}. They build up a topic model on the
source side and target side respectively, and learn a topic-specific
language model based on the target side by estimating the
maximum-likelihood. To smooth the sharply distributed probabilities,
back-off probabilities are also considered as follows:
\begin{align}
p(w_i | w^{i-1}_{i-n+1}, k_e) = &\lambda_{w^{i-1}_{i-n+1}} p_{MLE}(w_i|w^{i-1}_{i-n+1}, k_e) \\
&+ (1- \lambda_{w^{i-1}_{i-n+1}})p_{MLE}(w_i|w^{i-1}_{i-n+2}, k_e)
\end{align}
where $\lambda$ is the normalization parameter, calculated by,
\begin{align}
\lambda_{w^{i-1}_{i-n+1}, k_e} = \frac{N_{1+}(w^{i-1}_{i-n-1},
  k_e)}{N_{1+}(w^{i-1}_{i-n-1}, k_e) + \sum_{w_i}c(w^i_{i-n+1}, k_e)}
\end{align}
where $N_{1+}(w^{i-1}_{i-n-1}, k_e)$ is the number of words following
$w^{i-1}_{i-n-1}$ in topic $k_e$, and $c(w^i_{i-n+1}, k_e)$ is the
count of n-gram $w^i_{i-n+1}$ in $k_e$.

During the decoding, since no target sentence is available, they extract the topics on the 
source side and project the source topic to the target side. 
The target probability is estimated as following:
\begin{align}
p(e) = \sum_{k_e} p(e|k_e) p(k_e) = \sum_{k_e} p(e|k_e) \cdot \sum_{k_f} p(k_e|k_f) p (k_f)
\end{align}
where $p(k_e|k_f)$ is the topic projection probability, estimated by
the co-occurrence of the source-side and the target-side topic assignment.


\section{Reordering with Topic Models}

In addition to translation models and languages models, a third
important component of a phrase-based \textsc{smt} systemt is 
reordering models, which learn how the order of words in the source
sentences influences the order of words in the target sentences and
how to make the translations in the right order.
The usefulness of topic models in reordering is less clear than their usefulness 
for domain adaptation of translation models
and language models, but it is nevertheless significant.
The primary advantage is that word order in different domains of the same
language may be different: \citet{Chen-2013} find that training corpora in different domains vary
significantly in their reordering characteristics for particular
phrase pairs.
As the example shown in
Table~\ref{tab:reorder-topic}~\citep{wang-14}, in an \underline{economy}
topic, the Chinese word \begin{CJK*}{UTF8}{gbsn}比\end{CJK*} is on the
  left of \begin{CJK*}{UTF8}{gbsn}五\end{CJK*}; but in a
    \underline{sports} topic, \begin{CJK*}{UTF8}{gbsn}比\end{CJK*} is
      on the right of \begin{CJK*}{UTF8}{gbsn}五\end{CJK*}. As a
        result, it is necessary to introduce domain knowledge
        to model such order variance, and topic models provide a good data-driven way to do so.

\begin{table}[!tp]
\begin{center}
\setlength\tabcolsep{3pt}
\begin{tabular}{c || c c} \hline
Topic & Type & Example \\ \hline \hline
\multirow{2}{*}{Economy} & Source & $\cdots$ \textbf{\begin{CJK*}{UTF8}{gbsn}比五\end{CJK*}} \begin{CJK*}{UTF8}{gbsn}月份下降\end{CJK*}$3.8\%$ $\cdots$\\
                     & Target & $\cdots$ down $3.8\%$ from May $\cdots$\\ \hline
\multirow{2}{*}{Sports} & Source & $\cdots$ \textbf{\begin{CJK*}{UTF8}{gbsn}五比\end{CJK*}}\begin{CJK*}{UTF8}{gbsn}一\end{CJK*}$3.8\%$ $\cdots$\\
                     & Target & $\cdots$ five to one $\cdots$\\ \hline
\end{tabular}
\caption{Topics influence the word orders: the Chinese words in bold
  are in different orders in different topics. (Example from
  \citet{wang-14})}
\label{tab:reorder-topic}
\end{center}
\end{table}

\citet{Xiong-2006} treat the
reordering problem as a classification with two labels: straight and
inverted between two consecutive blocks, and build up a maximum
entropy classification model as the reordering model.
\citet{Chen-2013} manually divide the training data into
multiple domains, instead of using automatic techniques such as topic
models.
\citet{wang-14} integrate two more types of
topic-based features into the reordering model, in additional to the
boundary word features used in \citet{Xiong-2006}. First, they choose
the topic with maximum probability in a document to be the
\textit{document topic feature} for that document. Besides, they also
use the topics of the content words that locate at the left and
rightmost positions on the source phrases as the \textit{word topic
  features} to capture topic-sensitive reordering patterns.

During the decoding process, \citet{Xiong-2006} infer the topic
distributions of the test documents first and then apply this proposed
topic-based reordering model as one sub-model to the log-linear maximum entropy model
to obtain the best translation:
\begin{align}
e_\texttt{best} = \texttt{argmax}_e \Big \{ \sum_{m=1}^M \lambda_m h_m(e,f) \Big \}
\end{align}
where $h_m(e,f)$ are the sub-models or features of the whole
log-linear model, $\lambda_m$ are their weights accordingly, which are
tuned on the development set.

This framework is very flexible and can encode any topic-based features.
Any multilingual topic models we have discussed so far can be
applied to extract better topics.



%The goal of a lexicalized reordering model is to learn the orientation
%probabilities for a given phrase pair with respect to the previous
%phrase pair and the following phrase pair~\citep{Chen-2013}. The
%orientation $o$ typically includes three types: monotone (M), swap (S)
%and discontinuous (D). The M orientation means the current phrase pair
%is immediately to the right of the previously translated phrase in the
%source sentence, S orientation occurs when the current phrase is
%immediately to the left of the previous phrase, and all other cases
%all belong to D orientation.
%The model is ``lexicalized'' because the estimated orientations depend on the words in both the
%previously translated phrase pair and the current phrase pair.
%
%Formally, the reordering model is defined to estimate the
%corresponding probabilities $p(o|f,e)$ using recursive MAP smoothing:
%\begin{align}
%p(o|f,e) &= \frac{c(o,f,e) + \alpha_f p(o|f) + \alpha_e p(o|e)}{c(f,e) + \alpha_f + \alpha_e} \\
%p(o|f) &= \frac{c(o,f) + \alpha_g p(o)}{c(f) + \alpha_g} \\
%p(o|e) &= \frac{c(o,e) + \alpha_g p(e)}{c(e) + \alpha_g} \\
%p(o) &= \frac{c(o) + \alpha_u/3}{c(\cdot) + \alpha_u}
%\end{align}
%where $c(o,f,e)$ is the orientation counts obtained from the word-aligned corpus;  the parameters $\alpha_f$, $\alpha_e$, $\alpha_g$ and $\alpha_u$ are learned by minimizing the perplexity of the resulting model on the held-out data.
%
%\citet{Chen-2013} find that training corpora in different domains vary
%significantly in their reordering characteristics for particular
%phrase pairs, thus they introduce a linear model for adapting reordering
%models. Given $N$ sub-corpora (or $N$ domain corpus), they
%first train a reordering model on each domain-specific corpus, and then define
%the global reordering model as a linear combination of the reordering
%models on each domain corpus:
%\begin{align}
%p(o|f,e) = \sum_{i=1}^{N} \alpha_i p_i(o|f,e)
%\end{align}
%where $p_i(o|f,e)$ is the reordering model on sub-corpus $i$, and the weight $\alpha_i$ are learned by maximizing the probability of phrase-pair orientations in the in-domain development set:
%\begin{align}
%\hat{\alpha} = \texttt{argmax}_{\alpha} \sum_{o,f,e} \tilde{p}(o,f,e) \log \sum_{i=1}^N \alpha_i p_i(o|f,e)
%\end{align}
%where $\tilde{p}(o,f,e)$, proportional to $c(o,f,e)$, is the empirical distribution of counts in the development set. \citet{Chen-2013} propose to smooth the in-domain sample and weight instances by document frequency to further improve the mixture adaptation. They demonstrate this adaptation significantly improve the statistical machine translation system.
%
%However, \citet{Chen-2013} manually divide the training data into
%multiple domains, instead of using automatic techniques such as topic
%models. \citet{wang-14} apply topic models to maximum entropy
% phrase reordering models \citep{Xiong-2006}.
%
%Under the ITG constraints~\citep{wu-1997}, three Bracketing
%Transduction Grammar (BTG) rules are used to constrain the translation
%and reordering:
%\begin{align}
%A &\rightarrow [A^1, A^2]\\
%A &\rightarrow <A^1, A^2>\\
%A &\rightarrow f/e
%\end{align}
%where $A$ is a block with a pair of source and target strings; $A^1$
%and $A^2$ are two consecutive blocks; the two rules are reordering
%rules which merge two blocks into a larger block in a straight or
%inverted order. Rule 3 translates a source phrase $f$ to a target
%phrase $e$ and generate a block $A$. These three rules applied until the whole source sentence is covered, generating a 
%hierarchical segmentation tree of the source sentence in the process.
%
%Based on this hierarchical setting, \citet{Xiong-2006} treat the
%reordering problem as a classification with two labels: straight and
%inverted between two consecutive blocks, and build up a maximum
%entropy classification model as the reordering model:
%\begin{align}
%p(o|c(A^1, A^2)) = \frac{\exp(\sum_i \theta_i f_i(o, c(A^1, A^2)))}{\sum_{o'} \exp(\sum_i \theta_i f_i(o', c(A^1, A^2)))}
%\end{align}
%where $\theta_i$ are the feature weights, $c(A^1, A^2)$ indicates the attributes of $A^1$ and $A^2$, and $f_i(o, c(A^1, A^2))$ are binary features defined as,
%\begin{align}
%f_i(o, c(A^1,A^2)) = \begin{cases}
%1, &\text{if $(o,c(A^1, A^2))$ satisfies certain condition} \\
%0, &\text{else}
%\end{cases}
%\end{align}
%
%This framework only uses the features extracted from the blocks
%instead of the whole block (in contrast to lexicalized reordering),
%thus it is flexible to reorder any blocks~\citep{Xiong-2006}.




\section{Beyond Domain Adaptation}

In addition to translation models, language models and reordering models,
there are also other modules of \textsc{smt}, such as word alignment,
where topic models have also been applied. \citet{zhao-06} present a
Bilingual topical admixture (BiTAM) to improve  word
alignment in \textsc{smt}. BiTAM assumes each document pair is an
admixture of topics, and the topics for each sentence pair within that
document pair are sampled from the same document-topic
distribution. Each topic also has a topic-specific translation
table. Therefore, the sentence-level word alignment and translations
are coupled by the hidden topics.  BiTAM captures the latent
topical structure and generalizes word alignments and translations via
topics shared across sentence pairs, thus the quality of the
alignments is improved.

Besides, coherence, which ties sentences of text into a meaningfully
connected structure~\citep{xiong-13}, is another important piece to
\textsc{smt}. \citet{xiong-13} introduce a topic-based coherence model
to improve the document translation quality. They learn the sentence
topic for source documents, based on which they predict the target
topic chain; they then incorporate the predicted target coherence
chain into the document translation decoding process.

In general, multilingual topic models obtain topics with high quality,
since different languages can complement each other to reduce topic
ambiguity. Many different approaches, have been explored to apply such
multilingual topic models to improve different pieces of statistical
machine translation. With such topic knowledge, the variations of
different languages can be better captured to make the translations
more natural and coherent.



\chapter{Introduction}
\label{c-intro} % a label for the chapter, to refer to it later

\section{Obtaining the required files}
\label{s-files} % another label

You will need the following four files: 
\texttt{now.cls}, 
\texttt{frontmatter.tex},
\texttt{now\_logo.eps}, and 
\texttt{essence\_logo.eps}.
You can get the set for your journal from the \now\ website and they
should be placed into the directory with your \LaTeX\ source. The
\texttt{frontmatter.tex} file is specific to each journal; in this sample we
use the frontmatter appropriate for \emph{Foundations and
Trends\textsuperscript{\textregistered}  in Optimization}.

\iffalse
\section{Creating the output PDF file}
You create a PDF file by running the following commands:
\begin{verbatim}
pdflatex paper.tex
bibtex paper.tex
pdflatex paper.tex
pdflatex paper.tex
\end{verbatim}
You need to run \texttt{pdflatex} twice only to ensure that all the references are
up to date. If you are using a package that requires the use of PostScript, like
\texttt{psfrag}, then you should instead use the following:
\begin{verbatim}
latex paper.tex
bibtex paper.tex
latex paper.tex
latex paper.tex
dvipdf paper.dvi
\end{verbatim}
We recommend using \texttt{pdflatex} unless you have a good reason not to.
If you are unfamiliar with any of this, see Lamport~\citep{Lam:94} or any
other standard \LaTeX\ references.
\fi

\chapter{Converting Your \LaTeX\ Source}
\label{c-converting}

The \now\ style files follow standard \LaTeX\ constructions as much as possible
and are based on \LaTeX's \texttt{book} class,
so if you are starting with a good \LaTeX\ file, it should be very easy
to convert to \now\ format.

You can refer to a bare bones document outline in Appendix~\ref{s-bare-bones}.
To convert your article to a \now\ book, do the following:

\begin{enumerate}
\item If your paper currently uses the class \texttt{article}, 
convert all \texttt{\textbackslash section}s to \texttt{\textbackslash chapter}s, 
\texttt{\textbackslash subsection}s to
\texttt{\textbackslash section}s, and \texttt{\textbackslash subsubsection}s 
to \texttt{\textbackslash subsection}s.

\item Insert \texttt{\textbackslash frontmatter}, \texttt{\textbackslash mainmatter},
and \texttt{\textbackslash backmatter} at the locations shown in
Appendix~\ref{s-bare-bones}.

\item Remove the following packages:
\texttt{amsthm},
\texttt{caption},
\texttt{emptypage},
\texttt{fancyhdr},
\texttt{fontenc},
\texttt{fullpage},
\texttt{geometry},
\texttt{graphicx},
\texttt{lastpage},
\texttt{lmodern},
\texttt{multicol},
\texttt{natbib}, and
\texttt{url}.
They are already included or will interfere with the output.

\item Remove all commands that modify anything to do with the page size,
spacing, breaks, fonts, pagination, or anything else related to the display of
the document. This includes \texttt{clearpage}, \texttt{newpage},
\texttt{vfill}, \texttt{vspace}, and so on.

\item Ensure the paper's abstract is defined in an \texttt{abstract}
environment and placed \emph{after} the \texttt{mainmatter} command, as shown
in Appendix~\ref{s-bare-bones}.

\item Use the bibliography style \texttt{plain} or \texttt{plainnat} as needed
for the journal you are writing for. If you are using \texttt{plainnat}
(author-year style), use the \texttt{citep} and \texttt{citet} commands from
\texttt{natbib} rather than the usual \texttt{cite}.

\item Change the document class to

\texttt{\textbackslash documentclass\{now\}}

to produce the book version or 

\texttt{\textbackslash documentclass[openany]\{now\}}

to produce the journal version of the paper.

\item Optionally, include an \texttt{acknowledgements} or \texttt{acknowledgments}
environment (either spelling works) after the final main chapter but before any
appendices, as shown in Appendix~\ref{s-bare-bones}.

\item Optionally, include any of the following commands as needed:

\begin{verbatim}
\volume{1}
\issue{3}
\pubyear{2013}
\isbn{XXX}
\doi{XXX}
\firstpage{80}
\lastpage{94}
\copyrightowner{N.~Parikh and S.~Boyd}
\end{verbatim}

These can be included before \texttt{\textbackslash maketitle}. Default values
will be used for these if they are not provided. Consult \now\ for the
appropriate values for your paper; you should provide the copyright owner in
the format given above.
\end{enumerate}


\section{Bibliography}

Use a BiB\TeX\ file as usual and set the bibliography style to
\texttt{plain} or \texttt{plainnat} as needed for the particular
journal you are writing for.

\section{Referring to your paper}

Your paper is being published in multiple formats (such as a book and a journal
article), so you should avoid terms such as ``book'', ``article'', and ``paper''
when referring to the work itself. Instead, please use terms such as ``monograph'',
``tutorial'', ``review'', or ``survey''.`` 

\section{Titles}

Capitalize the first letter of all words of chapter titles 
(except for filler words like `the' and `and', of course). 
You should capitalize words appearing in hyphenated constructions, as in
Interior-Point Methods.
For all other titles (sections, subsections, subsubsections, paragraphs),
capitalize only the first letter and any proper nouns, as in this document.

You may prefer a different capitalization scheme, for example, 
capitalizing all words in section titles as well. This is fine, but be sure that
you are absolutely consistent in your scheme.

\section{References}

When referring to sections (chapters, appendices)
using \texttt{\textbackslash ref}, refer to chapters
as Chapter~5 but sections as \S5.1. To produce the section symbol, use
\texttt{\textbackslash S}.  When referring to tables or figures, write
Table~7.1 or Figure~5.3. When referring to equations, write simply (3.4) or
Equation~3.4. The former can be more easily obtained using
\texttt{\textbackslash eqref\{\}} from the \texttt{amsmath} package (not
included by the \now\ class itself).

\section{Preface and other special chapters}

If you want to include a preface, it should be defined as follows:
\begin{quote}
\begin{verbatim}
\chapter*{Preface}
\addcontentsline{toc}{chapter}{Preface}
\markboth{\sffamily\slshape Preface}
  {\sffamily\slshape Preface}
\end{verbatim}
\end{quote}
This ensures that the preface appears correctly in the table of contents
and the running headings.

You can follow a similar procedure if you want to include additional
unnumbered chapters (\eg, a chapter on notation used in the paper),
though all such chapters should precede Chapter 1.

Unnumbered chapters should not include numbered sections. If you want
to break your preface into sections, use the starred versions of
\texttt{section}, \texttt{subsection}, \etc.

\section{Figures and plots}

Don't use the optional positioning commands in figures, tables, etc.
Make caption text consistent in syntax and style.

\section{Mathematics}
Be consistent in formatting equations and make sure they are consistent 
with the surrounding English syntax.
As an example: 
\begin{quote}
When $\theta >0$, we have
\[
\theta + \theta^2 > \theta.
\]
\end{quote}
Note the comma after the fragment `$\theta>0$' and the period after the displayed
inequality. The sentence starts with the English word `When' and ends with 
the displayed equation.
(By the way, many authors have adopted the convention that a 
sentence should always start in English, and not with a mathematical 
formula or equation.)

Be sure to distinguish mathematics from English, even in sub- and super-scripts.
For example, do not write $sin(\theta)$, which typesets `sin' as
three mathematical symbols next to each other; the correct version is $\sin(\theta)$.
As a more subtle example, consider 
\[
\theta_i \leq \theta^\mathrm{max}, \quad i=1, \ldots, K.
\]
Here the superscript `max' is correctly rendered in roman font, since it refers
to the English word `max' or `maximum'.  The wrong way is to write $\theta_i
\leq \theta^{max}$. 

In general, mathematics should be typeset using built-in mathematical commands like
\texttt{\textbackslash sin}, variables should be typeset in math mode, and
English in math mode should be typeset in standard roman font.  Combining
these, we get, for example, $\sin(x^n + x^\mathrm{max})$.

\section{Source code}
\label{s-formatting-source}

We recommend that you hard wrap your \LaTeX\ source to, say, 72 characters or
equivalent, and that you put display mathematical equations in a paragraph on their
own line. See the source code of this document itself for an example.
This is to make the source code both easier to read and easier to debug. 
If you have entire paragraphs on a single line, it can become difficult
to track down \LaTeX\ compilation errors when it complains about a given line.

\section{Making the output look good}
\label{s-never-do-this}

You may be tempted to fine-tune the \LaTeX\ source to attempt to make the 
output look good
by, say, forcing a line break or a page break at a good place, or by adding 
or subtracting vertical or horizontal space in the document.
Our advice on this is simple:

\begin{center}
\textbf{Never do this.}
\end{center}

Never adjust the \LaTeX\ source to make the output look better.
Keep your \LaTeX\ source completely free of commands that adjust the 
particular output using custom spacing or positioning commands.
Leave this task to \TeX\ and \LaTeX.

Of course, tables and figures that are clearly too large to fit in
the margins should be appropriately shrunk, either by reducing the font size or
by breaking up the table to use fewer columns. Another example is a long URL
which wreaks havoc on the typesetting; you can put these on a line by
themselves.

The \emph{only exception} to this important rule is on the very last pass, 
right before your article goes into production; see Appendix~\ref{c-fine-tuning}.

% end of main matter

\begin{acknowledgements}
\addcontentsline{toc}{chapter}{Acknowledgements}
We would like to thank Donald Knuth and Leslie Lamport for their work
on \TeX\ and \LaTeX, respectively, as well as the many others who
have developed the \LaTeX\ packages we rely on for our own work.
\end{acknowledgements}

\appendix

\chapter{Bare Bones File Outline}
\label{s-bare-bones}

\begin{verbatim}
\documentclass{now}

\title{Article Title}

\author{
Neal Parikh \\
Stanford University \\
npparikh@cs.stanford.edu
\and
Stephen Boyd \\
Stanford University \\
boyd@stanford.edu
}

\begin{document}

% \firstpage, \doi, etc. as needed

\frontmatter

\maketitle

\tableofcontents

\mainmatter

\begin{abstract}
The abstract goes here.
\end{abstract}

\chapter{Introduction}
This is the text for the first chapter.

\chapter{Conclusion}
The text for the last chapter goes here.

\begin{acknowledgements}
Don't forget to thank your mom, without whom your article
would not have been possible.
\end{acknowledgements}

\appendix

\chapter{First Appendix}
This is an appendix with many technical details, or
a summary of other needed material.

\backmatter

\bibliographystyle{plain} % or plainnat
\bibliography{mybibfile}
\end{verbatim}

\chapter{Sample Formatting}

In this chapter, we illustrate various environments that are available to the
author. See the source code of this document itself to see how to produce this
display. You can define your own additional environments using the
usual \texttt{amsthm} procedure, but you should not customize the theorem style;
the theorem style will automatically use the \now\ one.

\section{Theorem environments}
\label{s-theorem-envs}

\begin{definition}[contraction mapping]
    Let $(X,d)$ be a metric space. Then a map $T : X \to X$ is called a
    \emph{contraction mapping} on $X$ if there exists $q \in [0,1)$ such
    that
    \[
        d(T(x),T(y)) \leq q d(x,y)
    \]
    for all $x,~y$ in $X$.
\end{definition}

\begin{theorem}[Banach fixed point theorem]
Let $(X, d)$ be a non-empty complete metric space with a contraction mapping 
$T : X \to X$. Then $T$ admits a unique fixed-point $x^*$ in $X$ (\ie, $T(x^*) = x^*$).
Furthermore, $x^*$ can be found as follows: start with an arbitrary element $x_0$ in
X and define a sequence $x_n$ by $x_n = T(x_{n-1})$, then $x_n \to x^*$.
\end{theorem}

\begin{proof}
Lorem ipsum dolor sit amet, consectetur adipiscing elit. Mauris non metus et
lorem euismod ullamcorper vel at justo. Curabitur dignissim dui eget suscipit
facilisis. Sed odio dolor, laoreet at tempus auctor, rhoncus ac lorem. Etiam ac
nibh lobortis, vehicula urna lacinia, aliquam quam.
\end{proof}

\begin{lemma}
Lorem ipsum dolor sit amet, consectetur adipiscing elit. Mauris non metus et
lorem euismod ullamcorper vel at justo. Curabitur dignissim dui eget suscipit
facilisis. Etiam ac nibh lobortis, vehicula urna lacinia, aliquam quam.
\end{lemma}

\begin{corollary}
Lorem ipsum dolor sit amet, consectetur adipiscing elit. Mauris non metus et
lorem euismod ullamcorper vel at justo. Curabitur dignissim dui eget suscipit
facilisis. Etiam ac nibh lobortis, vehicula urna lacinia, aliquam quam.
\end{corollary}

\begin{remark}
Lorem ipsum dolor sit amet, consectetur adipiscing elit. Mauris non metus et
lorem euismod ullamcorper vel at justo. Curabitur dignissim dui eget suscipit
facilisis. Etiam ac nibh lobortis, vehicula urna lacinia, aliquam quam.
\end{remark}

\begin{proposition}
Lorem ipsum dolor sit amet, consectetur adipiscing elit. Mauris non metus et
lorem euismod ullamcorper vel at justo. Curabitur dignissim dui eget suscipit
facilisis. Etiam ac nibh lobortis, vehicula urna lacinia, aliquam quam.
\end{proposition}

\begin{example}
Lorem ipsum dolor sit amet, consectetur adipiscing elit. Mauris non metus et
lorem euismod ullamcorper vel at justo. Curabitur dignissim dui eget suscipit
facilisis. Etiam ac nibh lobortis, vehicula urna lacinia, aliquam quam.
\end{example}

\section{Internet addresses}

The \now\ class file includes the \texttt{url} package, so you should wrap
email and web addresses with \texttt{\textbackslash url\{\}}. This will
also make these links clickable in the PDF.

\chapter{Fine-Tuning}
\label{c-fine-tuning}

Remember that you should generally never fine-tune your \LaTeX\ source to make
the output look good; see \S\ref{s-never-do-this} on
page~\pageref{s-never-do-this}.  This appendix describes the \emph{only time}
when you can and should fine-tune your source.  Before proceeding, be
sure that these are the very last adjustments to be made to your \LaTeX\
source before the article appears.

\section{Positioning floats}

You may need to alter the positioning information for floating
environments such as tables and figures. In general, tables and figures with
captions should appear at the top of the page. Unless a figure is large enough
to take up most of a page, it should not appear on a page by itself. If you
want to force, say, two figures to appear one on top of each other on
their own page, you can use the following code or equivalent:
\begin{quote}
\begin{verbatim}
\begin{figure}[p]
\begin{center}
\includegraphics{figure_one}
\end{center}
\label{f-one}
\caption{Figure One.}
\begin{center}
\includegraphics{figure_two}
\end{center}
\label{f-two}
\caption{Figure Two.}
\end{figure}
\end{verbatim}
\end{quote}
An alternative is to use the \texttt{subfigure} package.

Again, you should generally \emph{not} modify the placement of
figures and tables. This should be done only when really necessary 
and then only in the last pass over a document, right before it goes
into production.

\section{Pagination}

By and large, pagination is very well-handled by default by
\TeX's typesetting algorithms. In some cases, however, the typesetting
engine requires some help to do the right thing. This can even involve
rewriting specific parts of the text of the article. 

Bringhurst suggests\footnote{We have slightly
edited the raw text for brevity.} the following guidelines:
\begin{quote}
    \emph{Avoid leaving the end of a hyphenated word, or any
    word shorter than four letters, as the last line of a paragraph.}

    \emph{Never begin a page with the last line of a paragraph.}

    The typographic terminology is telling. The stub-ends left when paragraphs
    \emph{end} on the \emph{first} line of a page are called \emph{widows}.
    They have a past but not a future, and they look foreshortened and
    forelorn.  It is the custom to give them one additional line for company.
    This rule is applied in close conjunction with the next.  
\end{quote}
Fixing such issues often involves rewriting some sentences in the relevant
paragraph or preceding paragraphs. They should typically not be addressed
with manual page breaks or other forced typesetting.

\section{Long chapter and section names}

If you have a very long chapter or section name, it may not appear nicely
in the table of contents, running heading, document body, or some subset of these.
It is possible to have different text appear in all three places if needed
using the following code:
\begin{quote}
\begin{verbatim}
\chapter[Table of Contents Name]{Body Text Name}
\chaptermark{Running Heading Name}
\end{verbatim}
\end{quote}
Sections can be handled similarly using the \texttt{sectionmark} command
instead of \texttt{chaptermark}.

For example, the full name should always appear in the table of contents, but
may need a manual line break to look good. For the running heading,
an abbreviated version of the title should be provided. The appearance of the long
title in the body may look fine with \LaTeX's default line breaking method
or may need a manual line break somewhere, possibly in a different place from
the contents listing.

Long titles for the article itself should be left as is, with no manual line
breaks introduced. The article title is used automatically in a
number of different places by the class file, and manual line breaks will
interfere with the output. If you have questions about how the title appears
in the front matter, please contact \now.

\backmatter  % references

\bibliographystyle{plainnat}
\bibliography{bib/journal-full,bib/jbg}

\end{document}
